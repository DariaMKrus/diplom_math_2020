\section{Введение}
В данной бакалаврской работе изучается система уравнений Навье-Стокса. Данная система уравнений датируется 1822г., когда 
Навье \footnote{Клод"=Луи Навье французский инженер и физик} впервые записал уравнение в частных производных для потока 
вязкой жидкости. Стокс \footnote{Джордж Стокс --- ирландский математик и физик} внес свой вклад в 1842 и 1843гг. Эйлер 
записал уравнения в частных производных для жидкости с нулевой вязкостью --- совершено невзкой в 1757г. Это уравнение 
тоже полезно, но большинство реаьных жидкостей, включая воду и воздух, является вязкими, поэтому Навье и Стокс 
моделировали уравнение Эйлера таким образом, чтобы учесть это свойство. Они вывели примерно одинаковые уравннения независимо
друг от друга, поэтому оно называется в честь них обоих. Навье сделал в процессе вывода несколько математических ошибок, но
получил верный ответ, а у Стокса с математикой все было в порядке, и именно поэтому мы знаем, что ответ Навье верен, несмотря 
на ошибку.

Несмотря на довольно долгие математические изучения данной системы уравнений (начиная с 1822г.), вопрос существования и 
гладкости решений для системы Навье-Стокса остался открытым до сих пор. В анализе решений данной системы заключается суть 
одной из семи "проблем тысячилетия", за решение которых Математический институт Клэя назначил огромную премию. Одним из 
главных и глобальных толчков при изучении данной системы было доказательство Жаном Лере в 1934г существования и в ряде 
случаев единственности слабых решений для данной системы. Данная бакалаврская работа посвящена как раз рассмотрению и
изучению понятия слабого решения для системы уравнений Навье-Стокса, а так же изучению единственности слабых решений 
в двумерном случае.
\clearpage

\section {Понятие слабого решения}
Пусть $\Omega$ --- ограниченная область в пространстве $R^n$, где $n=2, 3$, с достаточно гладкой границей $\partial\Omega$.
Рассмотрим начально"=краевую задачу для системы уравнений Навье"=Стокса

\begin{equation}\label{eq:5.1}
    \begin{gathered}
        \frac{\partial v}{\partial t}+\sum_{i=1}^nv_i\frac{\partial v}{\partial x_i}-\nu\Delta v+\nabla p=f;
    \end{gathered}
\end{equation}

\begin{equation}\label{eq:5.2}
    \begin{gathered}
        \operatorname{div} v=0;
    \end{gathered}
\end{equation}

\begin{equation}\label{eq:5.3}
    \begin{gathered}
        v|_{t=0}=v_0;
    \end{gathered}
\end{equation}

\begin{equation}\label{eq:5.4}
    \begin{gathered}
        v|_{(0, T)\times\partial\Omega}=0.
    \end{gathered}
\end{equation}
Здесь $v=(v_1(t, x),\ldots,v_n(t, x))$ --- вектор"=функция скорости движения частицы жидкости, $p=p(t, x)$ --- функция давления,
$f=f(t, x)$ --- вектор"=функция плотности внешних сил, $\nu>0$ --- коэффициент вязкости.
$\Delta v=(\Delta v_1,\ldots,\Delta v_n)$, \ $\Delta v_i= \frac{\partial^2 v_i}{\partial x_1^2}+\ldots+\frac{\partial^2 v_i}{\partial x_n^2}$;
$\operatorname{div} v= \frac{\partial v_1}{\partial x_1}+\ldots+\frac{\partial v_n}{\partial x_n}$;
$\nabla p=(\frac{\partial p}{\partial x_1},\ldots,\frac{\partial p}{\partial x_n})$.

Сформулируем определение сильного решения рассматриваемой задачи (\ref{eq:5.1})-(\ref{eq:5.4}).
Для этого введем необходимые функциональные пространства:

$L_p(\Omega)$ --- множество измеримых функций, суммируемых с $p$-ой степенью, где $1\le p< \infty$,
и нормой $\|v\|_{L_p(\Omega)}=(\int\limits_\Omega |v(x)|^pdx)^{1/p}$.

Пространство $L_{\infty}(\Omega)$ состоит из измеримых существенно ограниченных функций $v:\Omega\rightarrow R^n$. 
Функция $v:\Omega\rightarrow R^n$ называется существенной ограниченной, если существует число $C_1<\infty$, что $|v(x)|\le C_1$ при почти всех $x\in\Omega$.
Норма в $L_{\infty}(\Omega)$ задается
$\|v\|_{L_{\infty}(\Omega)}=ess \ \underset{x\in\Omega}{sup} |v(x)|$.

$W_p^m(\Omega)$ --- где $m\geqslant 1, \ p\geqslant 1$, пространство Соболева, состоящее из
функции, которые со своими обобщенными частными производными до порядка $m$ включительно принадлежат пространству $L_p(\Omega)$.\\
Норма в $W_p^m(\Omega)$ задается $\|v\|_{W_p^m(\Omega)}=\left(\sum\limits_{|\alpha|\leqslant m} \int\limits_\Omega |D^{\alpha}v(x)|^pdx\right)^{1/p}$.

$C_0^{\infty}(\Omega)$ --- пространство бесконечно дифференцируемых функций на $\Omega$ со значениями в $R^n$ и с компактным носителем,
содержащимся в $\Omega$.

$\mathcal{V}$ --- множество функций $v\in C_0^{\infty}(\Omega)$, таких что $\operatorname{div}v=0$;

$H$ --- замыкание $\mathcal{V}$ по норме пространства $L_2(\Omega)$;

$V$ --- замыкание $\mathcal{V}$ по норме пространства $W_1^1(\Omega)$;

$L_p(a,b;X)$ --- где $1\le p< \infty$ пространство суммируемых с $p$-ой степенью функций на $[a, b]$ со значениями в
банаховом пространстве $X$. Норма пространства $L_p(a, b;X)$ задается 
$\|v\|_{L_p(a, b;X)}=(\int\limits_0^T\|v(s)\|_X^p ds)^{1/p}.$

Через $L_{\infty}(a,b;X)$ будем обозначать множество всех измеримых существенно ограниченных функций $v:[a,b]\rightarrow X$.
Множество $L_{\infty}(a,b;X)$ является банаховым пространством относительно нормы $\|v\|_{L_{\infty}(a,b;X)}=ess \ \underset{x\in\Omega}{sup} \|v(s)\|_X$.

Будем обозначать $E^*$ сопряженное пространство к пространству $E$.

$<f,\varphi>$ --- обозначим действие функционала $f$ из $E^*$ на элемент $\varphi$ из $E$.

$C_i$ --- будем обозначать положительные константы.

Пусть $f$ и $v_0$ --- заданные функции,
где $f\in L_2(0, T; L_2(\Omega))$ и $v_0\in V$.

\begin{definition}
    Сильным решением начально"=краевой задачи (\ref{eq:5.1})-(\ref{eq:5.4}) называется пара функций $v\in L_2(0, T; L_2(\Omega))$ и
    $p\in L_2(0, T; L_2(\Omega))$, удовлетворяющих следующим условиям:
\end{definition}

\begin{enumerate} 
    \item обобщенные частные производные функций, содержащихся в равенствах (\ref{eq:5.1})-(\ref{eq:5.2}), принадлежат пространству $L_2(0, T; L_2(\Omega))$;
    \item при подстановке функций уравнения (\ref{eq:5.1})-(\ref{eq:5.2}) обращаются в равенства в пространстве $L_2(0, T; L_2(\Omega))$;
    \item функция $v$ удовлетворяет начальному условию (\ref{eq:5.3}) и граничному условию (\ref{eq:5.4}).
\end{enumerate}

Введем понятие слабого решения. Для этого пусть $v$ и $p$ --- сильное решение задач (\ref{eq:5.1})-(\ref{eq:5.4}).

Чтобы обеспечить понимание определения слабого решения, мы временно предположим, что $v=v(t, x)$ и $p=p(t, x)$ являются, фактически, гладкими решениями задачи.
Сопоставим функции $v$ отображение $v:[0, T]\rightarrow W^1_2(\Omega)$, определенное по формуле $$[v(t)](x)=v(t, x), t\in[0, T], x\in\Omega.$$
Другими словами, $v$ рассматривается не как функции переменных $t$ и $x$, а как функция переменной $t$, определенная на отрезке
$[0, T]$ и принимающая значения в функциональном пространстве $W^1_2(\Omega)$.

Аналогично определим $p:[0, T]\rightarrow L_2(\Omega)$ по формуле $$[p(t)](x)=p(t, x), t\in[0, T], x\in\Omega$$
и функцию $f:[0, T]\rightarrow L_2(\Omega)$ по формуле $$[f(t)](x)=f(t, x), t\in[0, T], x\in\Omega.$$\\

Умножая равенство (\ref{eq:5.1}) при фиксированных значениях $t\in [0, T]$ на функцию $\varphi (x)\in V$ скалярно в $L_2(\Omega)$, получим
$$\int\limits_\Omega \frac{\partial v}{\partial t}\ \varphi dx+\sum_{i=1}^n\int\limits_\Omega v_i\frac{\partial v}{\partial x_i}\ \varphi dx-$$
$$-\nu \sum_{i, j=1}^n\int\limits_\Omega\frac{\partial^2 v_j}{\partial x^{2}_i} \varphi_j dx+
\sum_{i=1}^n\int\limits_\Omega \frac{\partial p}{\partial x_i}\ \varphi_idx=\int\limits_\Omega f\varphi dx.$$
Выполним преобразования слагаемых, связанные с интегрированием по частям
\footnote{Теорема Грина (теорема об интегрировании по частям функций нескольких переменных)\\
 Пусть $\Omega$ --- область с липшецевой границей и пусть $f(x)=f(x_1,...,x_n)$ и $g(x)=g(x_1,...,x_n)\in W_2^1(\Omega).$
 Тогда выполнено соотношение:
 $$\int\limits_\Omega\frac{\partial f}{\partial x_i} gdx=\int\limits_{\partial\Omega} fgv_i \ ds-\int\limits_\Omega f \frac{\partial g}{\partial x_i} dx,$$
 где $v_i$ --- $i-$ая координата единичного вектора внешней нормали.
}
$$-\nu\int\limits_\Omega \sum_{i,j=1}^{n}\frac{\partial^2 v_j}{\partial x^{2}_i}\varphi_j dx=
\nu\int\limits_\Omega\sum_{i, j=1}^{n}\frac{\partial v_j}{\partial x_i}\frac{\partial \varphi_j}{\partial x_i}=\nu\int\limits_\Omega \nabla v:\nabla\varphi dx.$$
Здесь символ <<:>> обозначает покомпонентное матричное произведение, т. е. для $ C=(c_{ij}), D=(d_{ij}), i, j=1,\ldots,m,$ имеем $C:D=\sum\limits_{i,j=1}^{m} c_{ij}d_{ij}.$
$$\sum_{i=1}^{n}\int\limits_\Omega \frac{\partial p}{\partial x_i}\varphi_i dx
=-\sum_{i=1}^{n}\int\limits_\Omega p\frac{\partial \varphi_i}{\partial x_i}dx= \int\limits_\Omega p \ div \ \varphi \ dx=0;$$

$$\sum_{i=1}^{n}\int\limits_\Omega v_i\frac{\partial v}{\partial x_i}\varphi dx=-\sum_{i=1}^{n}\int\limits_\Omega v\frac{\partial}{\partial x_i}(v_i\varphi)dx=$$
$$=-\sum_{i=1}^{n}\int\limits_\Omega v\frac{\partial v_i}{\partial x_i}\varphi dx-\sum_{i=1}^{n}\int\limits_\Omega vv_i\frac{\partial\varphi}{\partial x_i} dx=$$
$$=-\int\limits_\Omega v \ \varphi \ div \ v \ dx-\sum_{i=1}^{n}\int\limits_\Omega v_i v\frac{\partial\varphi}{\partial x_i}dx=
-\sum_{i=1}^{n}\int\limits_\Omega v_i v\frac{\partial\varphi}{\partial x_i}dx.$$
Таким образом, приходим к равенству
\begin{equation}\label{eq:5.5}
    \begin{gathered}
        \frac{d}{dt}\int\limits_\Omega v \varphi dx-\sum_{i=1}^{n}\int\limits_\Omega v_i v
        \frac{\partial \varphi }{\partial x_i}dx + v\int\limits_\Omega \nabla v :\nabla\varphi dx
        =\int\limits_\Omega f\varphi dx.
    \end{gathered}
\end{equation}

Заметим, что равенство (\ref{eq:5.5}) может выполняться и при более слабых требованиях на функцию $v(t, x)$.
Покажем, что достаточно предполагать, что $v\in L_2(0, T;V)$ для того, чтобы каждый интеграл, входящий в равенство (\ref{eq:5.5}), имел смысл.
\clearpage

В силу теоремы вложений Соболева\footnote{ Пусть $p_1\ge p$ и $\ae(W_{p_1}^{m_1}(\Omega))>\ae(W_{p}^{m}(\Omega)),$ то пространство $W_{p}^{m}(\Omega)$ компактно
вложено в пространоство $W_{p_1}^{m_1}(\Omega)$, где $\ae(W_{p}^{m}(\Omega))=\frac{n}{p}-m.$} вложение $W^{1}_{2}(\Omega)\subset L_4(\Omega)$ непрерывно при $n \leqslant 4$.
Поэтому, так как $V\subset W^{1}_{2}(\Omega)$, то $v_i(t, x)v(t, x)\in L_2(\Omega)$ и $v_i(t, x)v(t, x)\frac{\partial \varphi }{\partial x_i} \in L_1(\Omega)$
при каждом фиксированном значении $t$. Следовательно, интеграл $\sum\limits_{i=1}^{n}\int\limits_\Omega v_iv \frac{\partial\varphi}{\partial x_i}dx$ определен.

Кроме того, это слагаемое определяет линейный непрерывный функционал на $V$. Обозначим этот функционал через $K(v)$:
$$\langle K(v),\varphi \rangle = \sum_{i=1}^{n}\int\limits_\Omega v_i v \frac{\partial \varphi}{\partial x_i}dx.$$

Отметим, что $\int\limits_\Omega v \varphi dx\in L_2(0, T)$ и производная в выражении \linebreak
$\frac{d}{dt}\int\limits_\Omega v \varphi dx$
понимается в смысле распределений на интервале $(0, T)$. Поэтому равенство (\ref{eq:5.5}) выполняется в смысле распределений.
Все слагаемые равенства, исключая первое, принадлежат пространству $L_1(0, T)$, поэтому $\frac{d}{dt}\int\limits_\Omega v \varphi dx\in L_1(0, T)$
и равенство (\ref{eq:5.5}) выполняется для почти всех значений $t\in (0, T)$.

Подводя итог рассуждениям, приходим к следующему определению слабого решения.
\begin{definition}
    Пусть $f\in L_2(0,T;L_2(\Omega))$ и $v_0\in H$. Слабым решением задачи (\ref{eq:5.1})-(\ref{eq:5.4}) называется функция $v\in L_2(0, T;V)$,
    удовлетворяющая для всех $\varphi \in V$ и для почти всех значений $t \in (0, T)$ равенству
    \begin{equation}\label{eq:5.6}
        \begin{gathered}
            \frac{d}{dt}\int\limits_\Omega v \varphi dx-\sum_{i=1}^{n}\int\limits_\Omega v_i v
            \frac{\partial \varphi }{\partial x_i}dx + \nu\int\limits_\Omega \nabla v :\nabla\varphi dx=
            \int\limits_\Omega f \varphi dx
        \end{gathered}
    \end{equation}
    и условию
    \begin{equation}\label{eq:5.7}
        \begin{gathered}
            v(0) = v_0.
        \end{gathered}
    \end{equation}
\end{definition}

Выше показано, что равенство (\ref{eq:5.6}) корректно для $v \in L_2(0, T; V)$ и если $(v, p)$ сильное решение задачи (\ref{eq:5.1})-(\ref{eq:5.4}),
то $v$ является слабым решением. Поэтому задачу о поиске сильных решений заменим задачей об исследовании слабых решений.

Заметим, однако, что для функции $v \in L_2(0, T; V)$ условие (\ref{eq:5.7}) не имеет смысла, так как не определено значение функции $v(t)$ в каждой точке
$t \in (0, T)$. Покажем, что функция $v(t)$, удовлетворяющая равенству (\ref{eq:5.6}), является непрерывной на $[0, T]$ со значениями в $V^\ast$ и слабо
непрерывной со значениями в $H$. Поэтому равенство (\ref{eq:5.7}) имеет смысл и определение слабого решения корректно.

Преобразуем равенство (\ref{eq:5.6}). Скалярное произведение $(v(t), \varphi)_{L_2(\Omega)}$ определяет линейный непрерывный функционал на $H$, а следовательно,
элемент из $H^\ast$. Учитывая отождествление $H \equiv H^\ast$ и цепочку вложений $V \subset H \subset H^\ast \subset V^\ast$,
элемент $v(t)$ можно рассматривать как функционал на $V$, действие которого на функцию $\varphi \in V$ определяется равенством 
$\langle v(t),\varphi \rangle = (v(t),\varphi)_{L_2(\Omega)}$.Тогда можно считать, что функция $v(t)$ на $[0, T]$ принимает значения в $V^\ast$ и
$$\frac{d}{dt}\int\limits_\Omega v \varphi dx=\frac{d}{dt}\langle v(t),\varphi \rangle.$$

С учетом введенных обозначений равенство (\ref{eq:5.6}) можно записать в виде
$$\frac{d}{dt} \langle v(t),\varphi \rangle - \nu\langle \Delta v(t), \varphi \rangle -
\langle K(v(t)),\varphi \rangle = \langle f(t),\varphi \rangle,$$
где $\Delta: V \rightarrow V^\ast$, обозначает оператор Лапласа, действующий по правилу
$\langle \Delta v(t),\varphi \rangle=-\int\limits_\Omega \nabla v:\nabla\varphi dx.$
Или можно (\ref{eq:5.6}) переписать в виде:
\begin{equation}\label{eq:5.8}
    \begin{gathered}
        \frac{d}{dt} \langle v(t),\varphi \rangle = \langle \nu\Delta v(t)+K(v(t))+f(t),\varphi \rangle.
    \end{gathered}
\end{equation}

Исследуем свойства операторов, входящих в правую часть равенства.
\begin{lemma} ~\\
    \begin{enumerate}
        \item Оператор $\Delta: L_2(0, T; V) \rightarrow L_2(0, T; V^\ast)$ линейный и\linebreak
        непрерывный, причем
        \begin{equation}\label{eq:5.9}
            \begin{gathered}
                \| \Delta v \|_{L_2(0, T; V^\ast)} = \| v \|_{L_2(0, T; V)}, \ \forall v \in L_2(0, T; V^\ast).
            \end{gathered}
        \end{equation}
        \item Оператор $K: L_2(0, T; V) \rightarrow L_1(0, T; V^\ast)$ непрерывен и справедлива оценка
        \begin{equation}\label{eq:5.10}
            \begin{gathered}
                \| K(v) \|_{L_1(0, T; V^\ast)} \leqslant C_2\| v \|^2_{L_2(0, T;V)}, \ \forall v \in L_2(0, T; V^\ast),
            \end{gathered}
        \end{equation}
    \end{enumerate}
    для некоторой константы $C_2$.
\end{lemma}

\begin{proof}
        Покажем, что оператор $\Delta : V\rightarrow V^*$ линейный. 
        Для этого возьмем $v$ и $u$, принадлежащая пространству $V$и применим к ним оператор Лапласа
        $$\Delta (\alpha v+\beta u)= \sum_{i,j=1}^{n} \frac{\partial^2}{\partial x_i^2} (\alpha v_i+\beta u_i)=
        \alpha \sum_{i,j=1}^{n} \frac{\partial^2 v_i}{\partial x_i^2}+\beta \sum_{i,j=1}^{n} \frac{\partial^2 u_i}{\partial x_i^2}=\alpha\Delta v+\beta\Delta u.$$
        

        Заметим, что оператор $\Delta : V\rightarrow V^*$ определяет изометрию пространств. Действительно:
        $$\| \Delta v \|_{V^*}=\underset{\varphi\in V}{sup} \frac{|\langle \Delta v,\varphi \rangle|}{\| \varphi\|_V}=\underset{\varphi\in V}{sup} 
        \frac{\bigg|\int\limits_{\Omega}\nabla v:\nabla\varphi dx\bigg|}{\| \varphi\|_V}\le
        \underset{\varphi\in V}{sup} \frac{\|\nabla v\|_{L_2(\Omega)}\|\nabla \varphi\|_{L_2(\Omega)}}{\|\varphi\|_V}\le$$
        $$\le\underset{\varphi\in V}{sup} \frac{\| v \|_{V} \| \varphi \|_{V}}{\| \varphi \|_{V}}=\| v \|_{V}.$$
        То есть $\| \Delta v \|_{V^*}\le\| v \|_{V}.$ С другой стороны, положим $\varphi=v:$\\
        $$|\langle \Delta v,v \rangle|=|\int\limits_{\Omega}\nabla v:\nabla v dx|=\|v\|_V^2.$$\\
        Применим неравенство Коши-Буняковского
        $$\|v\|_V^2=|\langle \Delta v,v\rangle|\le\|\Delta v\|_{V^*}\| v\|_{V}.$$
        Сократив на $\|v\|_V$, получим $\|\Delta v\|_{V^*}\ge\|v\|_V.$
        Следовательно получаем $\|\Delta v\|_{V^*}=\|v\|_V.$
        Заметим, что линейный ограниченный оператор является непрерывным.\\
        Отсюда для $v \in L_2(0, T; V)$ имеем
        $\| \Delta v \|_{V^\ast} = \| v(t) \|_V$ для почти всех\linebreak$t \in [0, T]$. Так как $\| v(t) \|_V \in L_2(0, T)$,
        то $\| \Delta v(t)\|_{V^*} \in L_2(0, T)$. Следовательно, $\Delta v \in L_2(0, T; V^\ast)$ и справедливо
        равенство (\ref{eq:5.9}). Таким образом, линейный оператор $\Delta$ определяет изометрию пространств $L_2(0, T; V)$ и $L_2(0, T; V^\ast)$.
        
        2) По определению оператора $K$ для любых $v,\varphi\in V$ действует по правилу
        $$\langle K(v),\varphi \rangle = \sum_{i,j=1}^{n}\int\limits_\Omega v_i v_j \frac{\partial \varphi_j}{\partial x_i}dx.$$
        Значит для любого $v\in V$ получим
        $$\bigg|\langle K(v),\varphi \rangle\bigg|\le\bigg|\int\limits_{\Omega}\sum_{i,j=1}^{n}v_i v_j \frac{\partial \varphi_j}{\partial x_i}dx\bigg|\le$$
        $$\le(\sum_{i,j=1}^{n}\int\limits_{\Omega}\bigg|v_i v_j\bigg|^2 dx)^{1/2}(\sum_{i,j=1}^{n}\int\limits_{\Omega}\bigg|\frac{\partial \varphi_j}{\partial x_i}\bigg|^2 dx)^{1/2}\le$$
        $$\le(\sum_{i=1}^{n}\int\limits_{\Omega}\bigg|v_i\bigg|^4 dx)^{1/4}(\sum_{j=1}^{n}\int\limits_{\Omega}\bigg|v_j\bigg|^4 dx)^{1/4}
        (\sum_{i,j=1}^{n}\int\limits_{\Omega}\bigg|\frac{\partial \varphi_j}{\partial x_i}\bigg|^2 dx)^{1/2}\le
        \|v\|^2_{L_4(\Omega)}\|\varphi\|_V.$$
        По теореме вложения Соболева вложение $V\in L_4(\Omega)^n$ непрерывно для $n\le 4$, поэтому, $\|v\|_{L_4(\Omega)^n}\le C_3\|v\|_V$ и, следовательно,
        $\|K(v)\|_{V^*}\le C_3^2\|v\|^2_V.$ Отсюда для $v\in L_2(0,T;V)$ имеем $K(v)\in L_1(0,T;V^*)$ и 
        $$\|K(v)_{L_1(0,T;V^*)}\|\le\int\limits_0^T\|K(v(t))\|_{V^*} dt\le C_3^2\int\limits_0^T\|v(t)\|_V^2 dt=C_3^2\|v(t)\|_{L_2(0,T;V)}^2.$$
        Докажем непрерывность оператора $K$. Для любых функций\linebreak
        $v,u\in L_2(0,T;V)$ справедлива оценка
        $$\int\limits_0^T\|K(v)-K(u)\|_{V^*} dt\le\int\limits_0^T(\sum_{i,j=1}^{n}\int\limits_\Omega (v_iv_j-u_iu_j)^2dx)^{1/2} dt\le$$
        $$\le\int\limits_0^T(\sum_{i,j=1}^{n}\int\limits_\Omega (v_i(v_j-u_j)+(v_i-u_i)u_j)^2dx)^{1/2}dt\le$$
        $$\le\sqrt2\sum_{i,j=1}^{n}\int\limits_0^T (\int\limits_{\Omega}v^2_i(v_j-u_j)^2dx)^{1/2}+
        (\int\limits_{\Omega}(v_i-u_i)^2u_j dx)^{1/2}\le$$
        $$\le\sqrt2\sum_{i,j=1}^{n}\int\limits_0^T(\|v_i\|_{L_4(\Omega)}\|v_j-u_j\|_{L_4(\Omega)}
        +\|v_i-u_i\|_{L_4(\Omega)}\|u_j\|_{L_4(\Omega)})dt\le$$
        $$\le C_2(\|v\|_{L_2(0,T;L_4(\Omega))}+\|u\|_{L_2(0,T;L_4(\Omega))})\|v-u\|_{L_2(0,T;L_4(\Omega))}.$$
        Отсюда, если $\|v-u\|_{L_2(0,T;L_4(\Omega))}\rightarrow 0$, то $\|K(v)-K(u)\|_{L_1(0,T;V)}\rightarrow 0$. Поэтому отображние $K$ непрерывно.
        
\end{proof}

По утверждению леммы $\nu\Delta v \in L_2(0, T; V^\ast), K(v) \in L_1(0, T; V^\ast)$, поэтому $\nu\Delta v(t) + K(v(t)) + f(t) \in L_1(0, T; V^\ast)$.
Тогда из равенства (\ref{eq:5.8}) и \linebreak теоремы \footnote{Теорема Пусть $X$ --- банахово пространство с сопряженным $X^*$ и функции $u,g$  принадлежат пространству
$L_1(a,b,X).$ Тогда следующие три условия эквивалентны:\\
(а) функция $u(t)$ почти всюду равна первообразной от $g(t)$ и для п.в. $t\in[a,b]$ $$u(t)=\xi+\int\limits_a^b g(s)ds, \xi\in X;$$\\
(b) для каждой пробной функции $\eta\in\mathcal{D}(a,b)$ $$\int\limits_a^b u(t)\eta'(t)dt= -\int\limits_a^b g(t)\eta(t)dt;$$\\
(c) для кажого $\phi\in X^*$ $$\frac{d}{dt}\langle \phi, u(t) \rangle = \langle \phi, g(t) \rangle$$ в смысле скалярных распределений на $(a,b).$
Если условия (а)-(с) выполнены, то $u$, в частности, почти всюду равна некоторой непрерывной функции.} следует:
\begin{enumerate}
    \item что функция $v(t)$ имеет суммируемую производную $v^\prime(t);$
    \item в силу равенства $\frac{d}{dt}\langle \phi, u(t) \rangle = \langle \phi, g(t) \rangle;$
    $$\frac{d}{dt} \langle v(t), \varphi \rangle = \langle v'(t), \varphi \rangle;$$
    \item равенство $\frac{d}{dt}\langle v(t), \varphi \rangle=\langle \Delta v(t)+K(v(t))+f(t),\varphi\rangle$ можно записать в виде
    $$v^\prime(t)=\nu\Delta v(t)+K(v(t))+f(t).$$
\end{enumerate}

Подводя итог рассуждений, отметим, что так как $v^\prime(t) \in L_1(0, T; V^\ast)$, в силу леммы 1.2
\begin{lemma}
Для $p_0\ge 1, \ p_1\ge 1$ имеет место вложение\linebreak
$W_{p_0,p_1}=\{v\in L_{p_0}(a,b,X_0), \ v_1\in L_{p_0}(a,b,X_0)\}\subset C([a,b],X_1)$
и это вложение непрерывно.
\end{lemma}
\noindent
поэтому функция $v(t)$ непрерывна на отрезке $[0, T]$ со значениями в $V^\ast$. 
Кроме того, в виду леммы 1.3,
\begin{lemma}
Пусть $X$ и $Y$ --- банаховы пространства, такие, что $X$ --- рефлективно и вложение $X\subset Y$ непрерывно.
Если функция $v\in L_{\infty}(a,b;X)$ слабо непрерывна как функция со значениями в $Y$,то и слабо непрерывна и как функция со значениями в $X$.
\end{lemma}
\noindent
если функция $v\in L_{\infty} (0,T;H)$, то эта функция слабо непрерывна со значениями в $H$. Поэтому начальное условие (\ref{eq:5.7}) имеет смысл.

Переформулируем понятие слабого решения
\begin{definition}
    Пусть $n \le 4, \ f \in L_2(0, T; L_2(\Omega))$ и $v \in H$. Слабым решением задачи (\ref{eq:5.1})-(\ref{eq:5.4}) называется функция
    $v \in L_2(0, T; V)\cap L_{\infty}(0,T;H)$ и условию$ v^{\prime}\in L_1(0, T; V^\ast)$, удовлетворяющая при почти всех значений $t \in (0, T)$ равенству 
    \begin{equation}\label{eq:5.11}
        \begin{gathered}
            v^\prime(t) - \nu\Delta v(t) - K(v(t)) = f(t)
        \end{gathered}
    \end{equation}
      и начальному условию
    \begin{equation}\label{eq:5.12}
        \begin{gathered}
            v(0) = v_0.
        \end{gathered}
    \end{equation}
\end{definition}
Для системы уравнений Навье-Стокса (\ref{eq:5.1})-(\ref{eq:5.4}) Ж. Лере в 1934 был получен следующий результат:
\begin{theorem}
Пусть $n=2,3$. Для каждой функции $f\in L_2(0,T;V^\ast)$ и $v_0\in H$ начально-краевая задача (\ref{eq:5.1})-(\ref{eq:5.4})
имеет хотя бы одно слабое решение $v.$
\end{theorem}
\clearpage

\section {Единственность слабого решения }
В этом разделе мы рассмотрим вопрос о единственности слабого решения начально-краевой задачи (\ref{eq:5.1})-(\ref{eq:5.4}) для эволюционной
системы уравнений Навье"=Стокса. Покажем, что в случае $\Omega \subset R^2$ слабое решение начально-краевой задачи единственно.
Однако для размерности $n > 2$ аналогичное утверждение неверно. Примером, показывающим, что слабое решение задачи не единственно, служит результат
о бифуркации решений, содержащийся, например, в [1, гл.II, \S 4,п.4.4].

Сформулируем и докажем утверждение о единственности слабого решения в случае $n = 2$.

\begin{theorem}
    Пусть $\Omega$ ограниченная область в $R^2$ с достаточно гладкой границей $\partial\Omega$. Тогда слабое решение $v$  решение
    задачи (\ref{eq:5.1})-(\ref{eq:5.4}) единственно.
\end{theorem}
\begin{proof}
    Покажем единственность слабого решения. Предположим, что $u$ и $v$ -- слабые решения начально-краевой задачи (\ref{eq:5.1})-(\ref{eq:5.4}).
    Значит, для этих функций выполнено операторное равентво (\ref{eq:5.11}). Рассмотрим разность полученных равенств. 
    Для разности $w = v - u$ получим равенство
    $$w^\prime - \nu\Delta w(t) - K(v(t)) + K(u(t)) = 0$$
    Применим функционалы, стоящие в равенстве, к функции $w(t)$
    \begin{equation}\label{eq:5.36}
        \begin{gathered}
            \frac{1}{2}\frac{d}{dt}\int\limits_\Omega w(t, x)w(t, x)dt+\nu\int\limits_{\Omega} \nabla w(t,x):\nabla w(t,x)dx= \\
            =\sum_{i=1}^{2}\int\limits_\Omega v_i(t,x) v(t, x)\frac{\partial w(t, x)}{\partial x_i}dx-\sum_{i=1}^{2}\int\limits_\Omega u_i(t,x)u(t, x)\frac{\partial w(t, x)}{\partial x_i}dx.
        \end{gathered}
    \end{equation}
    Оценим правую часть полученного равенства.
    $$\sum_{i=1}^{2}\int\limits_\Omega v_i(t,x) v(t, x)\frac{\partial w(t, x)}{\partial x_i}dx-
    \sum_{i=1}^{2}\int\limits_\Omega u_i(t,x)u(t, x)\frac{\partial w(t, x)}{\partial x_i}dx=$$
    $$=\sum_{i=1}^{2}\bigg[\int\limits_\Omega v_i(t,x)v(t, x)\frac{\partial w(t, x)}{\partial x_i}dx-
    \int\limits_\Omega u_i(t,x)v(t, x)\frac{\partial w(t, x)}{\partial x_i}dx+$$
    $$+\int\limits_\Omega u_i(t,x) v(t, x)\frac{\partial w(t, x)}{\partial x_i}dx-
    \int\limits_\Omega u_i(t,x) u(t, x)\frac{\partial w(t, x)}{\partial x_i}dx\bigg]=$$
    $$=\sum_{i=1}^{2}\bigg[\int\limits_\Omega v(t, x)\frac{\partial w(t, x)}{\partial x_i}(v_i(t,x)-u_i(t,x))dx+$$
    $$+\int\limits_\Omega u_i(t,x) \frac{\partial w(t, x)}{\partial x_i}(v(t,x)-u(t,x))dx\bigg]=
    \sum_{i=1}^{2}\bigg[\int\limits_\Omega w_i(t,x)v(t,x)\frac{\partial w(t, x)}{\partial x_i}+$$
    $$+\int\limits_\Omega w(t,x)u_i(t,x)\frac{\partial w(t, x)}{\partial x_i}\bigg]dx.$$
    Используем интегрирование по частям для вычисления первого интеграла
    $$\sum_{i=1}^{2}\int\limits_\Omega w_i(t, x)v(t, x)\frac{\partial w(t, x)}{\partial x_i}dx=
    -\sum_{i=1}^{2}\int\limits_\Omega w_i(t, x)\frac{\partial v(t, x)}{\partial x_i}w(t, x)dx,$$
    так как $\sum\limits_{i=1}^{2}\frac{\partial w_i(t, x)}{\partial x_i} = \textrm{div } w(t, x)=0$. Используем интегрирование по частям для вычисления второго из интегралов
    $$\sum_{i=1}^{n=2}\int\limits_\Omega u_i(t,x)w(t, x)\frac{\partial w(t, x)}{\partial x_i}dx=
    \sum_{i=1}^{n=2}\int\limits_\Omega u_i(t,x)\cdot\frac{1}{2}\cdot\frac{\partial \mid w(t, x)\mid^2}{\partial x_i}dx=$$
    $$=-\frac{1}{2}\int\limits_\Omega\sum_{i=1}^{n=2}\frac{\partial u_i(t, x)}{\partial x_i}\cdot\mid w(t, x)\mid^2dx=0,$$
    так как $\sum\limits_{i=1}^{2}\frac{\partial u_i(t, x)}{\partial x_i} = \textrm{div} u(t, x)=0.$
    Отсюда и из равенства (\ref{eq:5.36}) получим
    $$\frac{1}{2}\frac{d}{dt}\int\limits_\Omega w(t, x)w(t, x)dx+\nu\int\limits_{\Omega}\nabla w(t,x):\nabla w(t,x)dx=$$
    $$=-\sum_{i=1}^{n=2}\int\limits_\Omega w_i(t, x)\frac{\partial v(t, x)}{\partial x_i}w(t, x)dx$$
    или
    \begin{equation}\label{eq:5.37}
        \begin{gathered}
            \frac{1}{2}\frac{d}{dt}\| w(t)\|_H^2+\nu\| w(t)\|_V^2\le
            \bigg|\sum_{i=1}^{n=2}\int\limits_\Omega w_i(t, x)\frac{\partial v(t, x)}{\partial x_i}w(t, x)dx\bigg|.
        \end{gathered}
    \end{equation}
    Оценим правую часть неравенства, используя неравенства Шварца:
    $$\bigg|\sum_{i=1}^{n=2}\int\limits_\Omega w_i(t, x)\frac{\partial v(t, x)}{\partial x_i}w(t, x)dx\bigg|=
    \bigg|\sum_{i,j=1}^{n=2}\int\limits_\Omega w_i(t, x)\frac{\partial v_j(t, x)}{\partial x_i}w_j(t, x)dx\bigg|\le$$
    $$\le\left(\sum_{i,j=1}^{n=2}\int\limits_\Omega \mid w_i(t, x)\mid^2|w_j(t, x)\mid^2dx \right)^{1/2}
    \cdot\left(\sum_{i,j=1}^{n=2}\int\limits_\Omega\bigg|\frac{\partial v_j(t, x)}{\partial x_i}\bigg|^2dx \right)^{1/2}\le$$
    $$\left(\sum_{i=1}^{n=2}\int\limits_\Omega \mid w_i(t, x)\mid^4dx\right)^{1/4}\left(\sum_{j=1}^{n=2}\int\limits_\Omega \mid w_j(t, x)\mid^4dx\right)^{1/4}\cdot$$
    $$\cdot\left(\sum_{i,j=1}^{n=2}\int\limits_\Omega\bigg|\frac{\partial v_j(t, x)}{\partial x_i} \bigg|^2dx \right)^{1/2}$$
    или
    $$\bigg|\sum_{i=1}^{n=2}\int\limits_\Omega w_i(t, x)\frac{\partial v(t, x)}{\partial x_i}w(t, x)dx\bigg|\le$$
    $$\le\sum_{iбо=1}^{n=2}\| w_i(t)\|_{L_4(\Omega)}\| w_j(t)\|_{L_4(\Omega)}\|\frac{\partial v_j(t)}{\partial x_i}\|_{L_2(\Omega)}.$$
    Учитывая, суммирование по повторяющимся индексам, получаем
    $$\bigg|\sum_{i=1}^{n=2}\int\limits_\Omega w_i(t, x)\frac{\partial v(t, x)}{\partial x_i}w(t, x)dx\bigg|\le\| w(t)\|_{L_4(\Omega)}^2\| v(t)\|_V.$$
    Применим неравенство О.А. Ладыженской и далее неравенство Коши \linebreak
    $a\cdot b=\varepsilon a^2+\frac{b^2}{4\varepsilon}c \ \varepsilon=\frac{\nu}{2^{1/2}}$, получим
    $$\bigg|\sum_{i=1}^{n=2}\int\limits_\Omega w_i(t, x)\frac{\partial v(t, x)}{\partial x_i}w(t, x)dx\bigg|\le$$
    $$\le 2^{1/2}\| w(t)\|_{L_2(\Omega)}\| w(t)\|_V\| v(t)\|_V\le$$
    $$\le \nu\| w(t)\|_V^2+\frac{1}{2^{3/2}\nu}\| w(t)\|_{L_2(\Omega)}^2\| v(t)\|_V.$$
    Подставляя полученное соотношение в неравенство (\ref{eq:5.37}), получаем
    $$\frac{1}{2}\frac{d}{dt}\| w(t)\|_H^2\le\frac{1}{2^{3/2}\nu}\| w(t)\|_H^2\| v(t)\|_V.$$
    Тогда из неравенства Гронуолла"=Беллмана [1, теорема 2б, глава IV, с.188] следует
    $$\| w(t)\|_H^2\le\| w(0)\|_H^2\exp\left(\int_0^t\frac{1}{2^{1/2}\nu}\| v(s)\|_Vds\right).$$
    Поскольку $w(0) = v(0)- u(0) = 0$, то из полученного выше неравенства приходим к выводу,
    что $w(t)=0$ для всех $t\in[0, T]$. Следовательно, $v=u$ и слабое решение задачи (\ref{eq:5.1})-(\ref{eq:5.4}) единственно.
\end{proof}


\clearpage
\section*{Список литературы}
\addcontentsline{toc}{section}{Список литературы}
\begin{enumerate} 
    \item Уравнения Навье-Стокса. Теория и численный анализ. / Р. Темам : Москва, 1987. — с. 409.
    \item  Аппроксимационно"=топологический переход к исследованию задач гидродинамики. 
    Система Навье-Стокса. / В.Г. Звягин , В.Т. Дмитриенко  Москва: Едиториал УРСС, 2004. — с. 112.
    \item  Математические вопросы / О.А. Ладыжская  М.: Наука, 1970. — с. 288.
    \item Некоторые методы решения нелинейных краевых задач / Ж.-Л. Лионс, М.: Мир, 1972г., — с. 587.
    \item Essai sur le mouvement d'un fluide visqueux emplissant l'space // \linebreak Leray J., Aeta. 
    Math 1934 V. 63 p. 193-248.
\end{enumerate}