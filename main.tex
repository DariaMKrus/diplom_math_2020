\section{Эволюционная система уравнений \\Навье-Стокса}
\subsection {Понятие слабого решения}
Пусть $\Omega$ --- ограниченная область в пространстве $R^n$, где $n=2, 3$, с достаточно гладкой границей $\partial\Omega$.
Рассмотрим начально"=краевую задачу для системы уравнений Навье"=Стокса

\begin{equation}\label{eq:5.1}
    \begin{gathered}
        \frac{\partial v}{\partial t}+\sum_{i=1}^nv_i\frac{\partial v}{\partial x_i}-\nu\Delta v+\nabla p=f;
    \end{gathered}
\end{equation}

\begin{equation}\label{eq:5.2}
    \begin{gathered}
        \operatorname{div} v=0;
    \end{gathered}
\end{equation}

\begin{equation}\label{eq:5.3}
    \begin{gathered}
        v|_{t=0}=v_0;
    \end{gathered}
\end{equation}

\begin{equation}\label{eq:5.4}
    \begin{gathered}
        v|_{(0, T)\times\partial\Omega}=0.
    \end{gathered}
\end{equation}

Здесь $v=(v_1(t, x),...,v_n(t, x))$ - вектор"=функция скорости движения частицы жидкости, $p=p(t, x)$ - функция давления,
$f=f(t, x)$ - вектор"=функция плотности внешних сил, $\nu>0$ - коэффициент вязкости.
$\Delta v=(\Delta v_1,...,\Delta v_n)$, \ $\Delta v_i= \frac{\partial^2 v_i}{\partial x_1^2}+...+\frac{\partial^2 v_i}{\partial x_n^2}$,
$\operatorname{div} v= \frac{\partial v}{\partial x_1}+...+\frac{\partial v}{\partial x_n}$,
$\nabla p=\frac{\partial p}{\partial x_1}+...+\frac{\partial p}{\partial x_n}$.
Сформулируем определение сильного решения рассматриваемой задачи (\ref{eq:5.1})-(\ref{eq:5.4}).
Для этого введем необходимые функциональные пространства:

$L_p(\Omega)$ - где $1\leqslant p\leqslant\infty$, множество измеримых функций, суммируемых с $p$-ой степенью,

$W_p^m(\Omega)$ - где $m\geqslant 1, \ p\geqslant 1$, пространство Соболева,
функции, которые со своими производными до порядка $m$ включительно принадлежат пространству $L_p(\Omega)$,

$C_0^{\infty}(\Omega)$ - бесконечно дифференцируемых функций на $\Omega$ со значениями $R^n(n=2, 3)$ и компактным носителем $\Omega$,

$\nu$ - множество ${u\in C_0^{\infty}(\Omega), div \ u=0}$

$H$-....,

$V$-...,

$L_p(a, b;X)$ - мы обозначим банаховы пространства непрерывных, слабо непрерывных и суммируемых с $p$-ой степенью функций на $[a, b]$ со значениями в
банаховом пространстве $X$,

$X^*$-...,

$<f,\varphi>$ - обозначим действие функционала $f$ из $V^{-\alpha}$ на элемент $\varphi$ из $V^{\alpha}$, $\alpha\geqslant 0$

Пусть $f$ и $v_0$"=заданные функции,
где $f\in L_2(0, T; L_2(\Omega))$ и $v_0\in V$.

\begin{definition}
    Сильным решением начально"=краевой задачи (\ref{eq:5.1})-(\ref{eq:5.4}) называется пара функций $v\in L_2(0, T; L_2(\Omega))$ и
    $p\in L_2(0, T; L_2(\Omega))$, удовлетворяющие следующим условиям:
\end{definition}

\begin{enumerate} 
    \item обобщенные частные производные функции, содержащиеся в равенствах (\ref{eq:5.1})-(\ref{eq:5.2}), принадлежат пространству $L_2(0, T; L_2(\Omega))$;
    \item при подстановке функций уравнения (\ref{eq:5.1})-(\ref{eq:5.2}) обращаются в равенства в пространстве $L_2(0, T; L_2(\Omega))$;
    \item функция $v$ удовлетворяет начальному условию (\ref{eq:5.3}) и граничному условию (\ref{eq:5.4}).
\end{enumerate}

Введем понятие слабого решения. Для этого пусть $v$ и $p$-сильное решение задач (\ref{eq:5.1})-(\ref{eq:5.4}).
Умножим равенство (\ref{eq:5.1}) на пробную функцию $\varphi (x)\in V$ скалярно в $L_2(\Omega)$

Пусть $(v, p)$-сильное решение задачи (\ref{eq:5.1})-(\ref{eq:5.4}).
Чтобы обеспечить понимание определения слабого решения, мы временно предположим, что $v=v(t, x), p=p(t, x)$ являются, фактически, гладкими решениями задачи.
Сопоставим функции $v$ отображение $v:[0, T]\rightarrow W^1_2(\Omega)$, определенное по формуле $$[v(t)](x)=v(t, x), t\in[0, T], x\in\Omega.$$
Другими словами, $v$ рассматривается не как функции переменных $t$ и $x$, а как функция переменной $t$, определенная на отрезке
$[0, T]$ и принимающая значения в функциональном пространстве $W^1_2(\Omega)$.

Аналогично определим $p:[0, T]\rightarrow L_2(\Omega)$ по формуле $$[p(t)](x)=p(t, x), t\in[0, T], x\in\Omega$$
и функцию $f:[0, T]\rightarrow L_2(\Omega)$ по формуле $$[f(t)](x)=f(t, x), t\in[0, T], x\in\Omega.$$
Умножая равенство (\ref{eq:5.1}) при фиксированных значениях $t\in [0, T]$ на функцию $\varphi (x)\in C_0^{\infty}(\Omega)$ скалярно в $L_2(\Omega)$, получим
$$\int\limits_\Omega \frac{\partial v}{\partial t}\ \varphi dx+\sum_{i=1}^n\int\limits_\Omega v_i\frac{\partial v}{\partial x_i}\ \varphi dx-$$
$$-\nu \sum_{i, / j=1}^n\int\limits_\Omega\frac{\partial^2 v_j}{\partial x^{2}_i} \varphi_j dx+
\sum_{i=1}^n\int\limits_\Omega \frac{\partial p}{\partial x_i}\ \varphi_idx=\int\limits_\Omega f\varphi dx.$$
Выполним преобразования слагаемых, связанные с интегрированием по частям,
$$-\nu\int\limits_\Omega \sum_{i=1}^{n}\frac{\partial^2 v_j}{\partial x^{2}_i}\varphi_j dx=
\nu\int\limits_\Omega\sum_{i, / j=1}^{n}\frac{\partial v_j}{\partial x_i}\frac{\partial \varphi_j}{\partial x_i}=\nu\int\limits_\Omega \nabla v:\nabla\varphi dx;$$
$$\sum_{i=1}^{n}\int\limits_\Omega \frac{\partial p}{\partial x_i}\varphi_i dx
=-\sum_{i=1}^{n}\int\limits_\Omega p\frac{\partial \varphi_i}{\partial x_i}dx= \int\limits_\Omega p div\varphi dx=0;$$

$$\sum_{i=1}^{n}\int\limits_\Omega v_i\frac{\partial v}{\partial x_i}\varphi dx=-\sum_{i=1}^{n}\int\limits_\Omega v\frac{\partial}{\partial x_i}(v_i\varphi)dx=$$
$$=-\sum_{i=1}^{n}\int\limits_\Omega v\frac{\partial v_i}{\partial x_i}\varphi dx-\sum_{i=1}^{n}\int\limits_\Omega vv_i\frac{\partial\varphi}{\partial x_i} dx=$$
$$=-\int\limits_\Omega v\varphi div v dx-\sum_{i=1}^{n}\int\limits_\Omega v_i v\frac{\partial\varphi}{\partial x_i}dx=
-\sum_{i=1}^{n}\int\limits_\Omega v_i v\frac{\partial\varphi}{\partial x_i}dx$$
приходим к равенству
\begin{equation}\label{eq:5.5}
    \begin{gathered}
        \frac{d}{dt}\int\limits_\Omega v \varphi dx-\sum_{i=1}^{n}\int\limits_\Omega v_i v
        \frac{\partial \varphi }{\partial x_i}dx + v\int\limits_\Omega \nabla v :\nabla\varphi dx
        =\int\limits_\Omega f\varphi dx,
    \end{gathered}
\end{equation}
Здесь символ <<:>> обозначает покомпонентное матричное произведение, т. е. для $я C=(c_{ij}), D=(d_{ij}), i, / j=1,...,m,$ имеем $C:D=\sum_{i, / j=1}^{m} c_{i, / j}d_{i, / j}$
Заметим, что тоже равенство (\ref{eq:5.5}) верно для любой функции $\varphi\in V$, так как каждая часть этого равенства линейно и непрерывно зависит от
$\varphi$ в $W^{1}_{2}(\Omega)$. Кроме того, равенство может выполняться и при более слабых требованиях на функцию $v(t, x)$.
Покажем, что достаточно предполагать, что $v\in L_2(0, T;V)$ для того, чтобы каждый интеграл, входящий в равенство (\ref{eq:5.5}), имеет смысл.

В силу теоремы вложений Соболева\footnote{Тут будет теорема вложений Соболева :)} вложение $W^{1}_{2}(\Omega)\subset L_4(\Omega)$ непрерывно при $n \leqslant 4$.
Поэтому, так как $V\subset W^{1}_{2}(\Omega)$, то $v_i(t, x)v(t, x)\in L_2(\Omega)$ и $v_i(t, x)v(t, x)\partial_i\varphi \in L_1(\Omega)$
при каждом фиксированном значении $t$. Следовательно, интеграл $\sum_{i=1}^{n}\int\limits_\Omega v_iv\cdot\frac{\partial\varphi}{\partial x_i}dx$ определен.

Кроме того, это слагаемое определяет линейный непрерывный функционал на $V$. Обозначим этот функционал через $K(v)$:
$$\langle K(v), \ \varphi \rangle = \sum_{i=1}^{n}\int\limits_\Omega v_i v \frac{\partial \varphi}{\partial x_i}dx.$$
Отметим, что $\int\limits_\Omega v \varphi dx\in L_2(0, T)$ и производная в выражении \linebreak
$\frac{d}{dt}\int\limits_\Omega v \varphi dx$
понимается в смысле распределений на интервале $(0, T)$. Поэтому равенство (\ref{eq:5.5}) выполняется в смысле распределений.
Все слагаемые равенства, исключая первое, принадлежат пространству $L_1(0, T)$, поэтому $\frac{d}{dt}\int\limits_\Omega v \varphi dx\in L_1(0, T)$
и равенство (\ref{eq:5.5}) выполняется для почти всех значений $t\in (0, T)$.

Подводя итог рассуждениям, приходим к следующему определению слабого решения.
\begin{definition}
    Пусть $f\in L_2(Q_T)$ и $v_0\in H$. Слабым решением задачи (\ref{eq:5.1})-(\ref{eq:5.4}) называется функция $v\in L_2(0, T;V)$ такая, что равенство
    \begin{equation}\label{eq:5.6}
        \begin{gathered}
            \frac{d}{dt}\int\limits_\Omega v \varphi dx-\sum_{i=1}^{n}\int\limits_\Omega v_i v
            \frac{\partial \varphi }{\partial x_i}dx + v\int\limits_\Omega \nabla v :\nabla\varphi dx=
            \int\limits_\Omega f \varphi dx
        \end{gathered}
    \end{equation}
    выполняется для почти всех значений $t \in (0, T)$ и выполнено для этой функции $v$ начальное условие
    \begin{equation}\label{eq:5.7}
        \begin{gathered}
            v(0) = v_0.
        \end{gathered}
    \end{equation}
\end{definition}

Выше показано, что равенство (\ref{eq:5.6}) корректно для $v \in L_2(0, T; V)$ и если $(v, p)$ сильное решение задачи (\ref{eq:5.1})-(\ref{eq:5.4}),
то $v$ является слабым решением. Поэтому задачу о поиске сильных решений заменим задачей об исследовании слабых решений.

Заметим, однако, что для функции $v \in L_2(0, T; V)$ условие (\ref{eq:5.7}) не имеет смысла, так как не определено значение функции $v(t)$ в каждой точке
$t \in (0, T)$. Покажем, что функция $v(t)$, удовлетворяющая равенству (\ref{eq:5.6}), является непрерывной на $[0, T]$ со значениями в $V^\ast$ и слабо
непрерывной со значениями в $H$. Поэтому равенство (\ref{eq:5.7}) имеет смысл и определение слабого решения корректно.

Преобразуем равенство (\ref{eq:5.6}). Скалярное произведение $(v(t), \varphi)_{L_2(\Omega)}$ определяет линейный непрерывный функционал на $H$, а следовательно,
элемент из $H^\ast$. Учитывая отождествление $H \equiv H^\ast$ и цепочку вложений $V \subset H \subset H^\ast \subset V^\ast$,
элемент $v(t)$ можно рассматривать как функционал на $V$, действие которого на функцию $\varphi \in V$ определяется равенством 
$\langle v(t), \ \varphi \rangle = (v(t), \ \varphi)_{L_2(\Omega)}$.Тогда можно считать, что функция $v(t)$ на $[0, T]$ принимает значения в $V^\ast$ и
$$\frac{d}{dt}\int\limits_\Omega v \varphi dx=\frac{d}{dt}\langle v(t), \ \varphi \rangle$$

С учетом введенных обозначений равенство (\ref{eq:5.6}) можно записать в виде
$$\frac{d}{dt} \langle v(t), \ \varphi \rangle - \nu\langle \Delta v(t) \varphi \rangle -
\langle K(v(t)), \ \varphi \rangle = \langle f(t), \ \varphi \rangle $$
где $\Delta: V \rightarrow V^\ast$, обозначает оператор Лапласа, действующий по правилу
$\langle \Delta v(t), \ \varphi \rangle=-\int\limits_\Omega \nabla v:\nabla\varphi dx.$
Или можно (\ref{eq:5.6}) переписать в виде:
\begin{equation}\label{eq:5.8}
    \begin{gathered}
        \frac{d}{dt} \langle v(t), \ \varphi \rangle = \langle \nu\Delta v(t)+K(v(t))+f(t), \ \varphi \rangle.
    \end{gathered}
\end{equation}

Исследуем свойства операторов, входящих в правую часть равенства.
\begin{lemma} ~\\
    \begin{enumerate}
        \item Оператор \linebreak $\Delta: L_2(0, T; V) \rightarrow L_2(0, T; V^\ast)$ линейный и непрерывный, причем
        \begin{equation}\label{eq:5.9}
            \begin{gathered}
                \parallel \Delta v \parallel_{L_2(0, T; V^\ast)} = \parallel v \parallel_{L_2(0, T; V)}, \ \forall v \in L_2(0, T; V^\ast).
            \end{gathered}
        \end{equation}
        \item Оператор $K: L_2(0, T; V) \rightarrow L_1(0, T; V^\ast)$ непрерывен и справедлива оценка
        \begin{equation}\label{eq:5.10}
            \begin{gathered}
                \parallel K(v) \parallel_{L_1(0, T; V^\ast)} \leqslant C_1\parallel v \parallel^2_{L_2(0, T;V)}, \ \forall v \in L_2(0, T; V^\ast),
            \end{gathered}
        \end{equation}
    \end{enumerate}
    для некоторой константы $C_1$.
\end{lemma}

\begin{proof} ~\\
    \begin{enumerate} 
        \item В главе 2 показано, что отображение $\Delta: V \rightarrow V^\ast$ линейно, непрерывно и определяет изометрию пространств.
        Следовательно, \linebreak $\parallel \Delta u \parallel_{V^\ast} = \parallel u \parallel_V$ для всех $u \in V$.Отсюда для $u \in L_2(0, T; V)$ имеем
        \linebreak $\parallel \Delta u \parallel_{V^\ast} = \parallel u(t) \parallel_V$ для почти всех $t \in [0, T]$.
        Так как $\parallel u(t) \parallel_V \in L_2(0, T)$,
        то $\parallel \Delta u(t)\parallel_V^\ast \in L_2(0, T)$. Следовательно, $\Delta u \in L_2(0, T; V^\ast)$ и справедливо
        равенство (\ref{eq:5.9}). Таким образом, линейный оператор $\Delta$ определяет изометрию пространств $L_2(0, T; V)$ и $L_2(0, T; V^\ast)$.
        \item По определению оператора $K$
        $$\langle K(t), u \rangle = \int\limits_\Omega u_i(x )u_j(x) \cdot \partial_i u_j(x) dx$$
        для $u, \ v \in V$. Повторяя рассуждения доказательства леммы 3.1 или используя оценки (3.10), получим
        $$\parallel K(v(t)) \parallel_{V^\ast} \le c_0 \parallel v(t) \parallel^2_V, \ \forall v \in V, \ t \in [0, T],$$
        с некоторой константой $c_0$.
        Отсюда для $v \in L_2(0, T; V)$ имеем $K(v) \in L_1(0, T; V^\ast)$ и
        $$\parallel K(v)\parallel_{L_1(0, T; V^*)}\le\int_0^T\parallel K(v(t))\parallel_{V^*}dt\le c_0\int_0^T\parallel v(t)\parallel_V^2dt=$$
        $$=c_0\parallel v(t)\parallel_{L_2(0, T; V)}^2$$
        Повторяя рассуждения доказательства леммы 3.1, получим, что для $v, \ u \in L_2(0, T; V)$ справедлива оценка
        $$\parallel K(v) \parallel_{V^*} \le \left(\int\limits_\Omega \sum_{i, j=1}^n \left(\frac{v_i v_j}{1+|v|^2}\right)^2 dx \right)^{1/2}
        \le \left(\int\limits_\Omega \sum_{i, j=1}^n (v_i v_j)^2 dx \right)^{1/2} \le$$
        $$ \le \left(\int\limits_\Omega \sum_{i=1}^n v_i^4(x)dx \right)^{1/4} \left(\int\limits_\Omega \sum_{j=1}^n v_j^4(x)dx \right)^{1/4} =
        \parallel v \parallel^2_{(L_4(\Omega))^n}.$$
        По теореме вложения Соболева $V \subset (L_4(\Omega))^n$ непрерывно для $n \le 4$, поэтому
        $$\parallel v \parallel^2_{L_4(\Omega)^n} \le c \parallel v \parallel^2_V \textrm{ и } \parallel K(v) \parallel_{V^*}
        \le c^2 \parallel v \parallel^2_V$$ для некоторой константы $c$.
        Из определения отображения $К(v)$ ясно, что для докозательства непрерывности $K(v)$ достаточно доказать непрерывность отображения
        $$\phi_{ij}: (L_4(\Omega))^n \rightarrow L_2(\Omega), \ \phi_{ij}(v) = \frac{v_i v_j}{1+|v|^2}, \ i,j = 1, 2, \dots n.$$
        Непрерывность каждого из этих отображений следует из теоремы М.А. Красносельского о непрерывности оператора суперпозиции и очевидной оценки
        $$|\phi_{ij}| \le |v_i v_j| \le \frac{1}{2}(|v_i|^2+|v_j|^2) \textrm{ для } v \in R^n.$$
        Таким образом, непрерывность отображения $K(v)$ установлена.
        \item Заметим, что для функции $\phi_{ij}$ удовлетворяют оценке
        % $$|\phi_{ij}| \le frac{1}$$
        Тогда из теоремы М.А. Красносельского о непрерывности оператора суперпозиции следует непрерывность отображения
        $$\phi_{ij}: (L_2(\Omega))^n \rightarrow L_2(\Omega), \ \phi_{ij}(v) = \frac{v_i v_j}{1+|v|^2}, \ i,j = 1, 2, \dots n$$
        а следовательно, и отображения $K(v): (L_2(\Omega))^n \rightarrow V^*$

        Вложение $V\subset (L_2(\Omega))^n$ вполне непрерывно в силу теоремы Релиха"=Кондрашова, поэтому отображение $K(v):V\rightarrow V^*$
        вполне непрерывно как суперпозиция вполне непрерывного оператора вложения $V\subset (L_2(\Omega))$ и непрерывного отображения
        $K(v):(L_2(\Omega))^n\rightarrow V^*$
        $$\int_0^T \parallel K(v)-K(u)\parallel_{V^*}dt\le\int_0^T (\sum_{i, j=1}^n\int\limits_\Omega (v_iv_j-u_iu_j)^2dx)^\frac{1}{2}\le$$
        $$\le\int_0^T(\sum_{i, / j=1}^n\int\limits_\Omega (v_i(v_j-u_j)+(v_i-u_i)u_j)dx)^\frac{1}{2}\le$$
        $$\le\sqrt{2}\sum_{i, / j=1}^n\int_0^T(\int\limits_\Omega v_i^2(v_j-u_j)^2dx)^\frac{1}{2}+(\int\limits_\Omega(v_i-u_i)^2u_j^2dx)^\frac{1}{2}dt\le$$
        $$\le\sqrt{2}\sum_{i, / j=1}^n\int_0^T(\parallel v_i(t)\parallel_{L_4(\Omega)}\parallel v_j(t)-u_j(t)\parallel_{L_4(\Omega)}+$$
        $$+\parallel v_i(t)-u_i(t)\parallel_{L_4(\Omega)}\parallel u_j(t)\parallel_{L_4(\Omega)})dt\le$$
        $$\le C_2(\parallel v \parallel_{L_2(0, T; L_4(\Omega))}+
        \parallel u \parallel_{L_2(0, T; L_4(\Omega))})\parallel v-u\parallel_{L_2(0, T; L_4(\Omega))}$$
        Кроме того, вложение $V \subset (L_4(\Omega))^n$, а следовательно, и вложение $L_2(0, T; V) \subset L_2(0, T; (L_4(\subset))^n)$
        непрерывны. Поэтому достаточно доказать непрерывность отображений
        $$\phi_{ij}: L_2(0, T; (L_4(\Omega))^n) \rightarrow L_1(0, T; L_2(\Omega)), \varphi_{ij}(v) = v_i v_j$$
        для $i, / j = 1, 2, \dots, n$.
        Для любых $u, \ v \in L_2(0, T; (L_4(\Omega))^n)$ с помощью неравенства Шварца получаем оценку
        $$\parallel\phi_{ij}(u)-\phi_{ij}(v)\parallel_{L_1(0, T; L_2(\Omega))}=\int_0^T(\int\limits_\Omega(v_iv_j-u_iu_j)^2dx)^\frac{1}{2}\le$$
        $$\le\int_0^T(\int\limits_\Omega v_i^2(v_j-u_j)^2dx)^\frac{1}{2}+(\int\limits_\Omega (v_i-u_i)^2u_j^2dx)^\frac{1}{2}dt\le$$
        $$\int_0^T(\parallel v_i(t)\parallel_{L_4(\Omega)}\parallel v_j(t)-u_j(t)\parallel_{L_4(\Omega)}+$$
        $$+\parallel v_i(t)-u_i(t)\parallel_{L_4(\Omega)}\parallel u_j(t)\parallel_{L_4(\Omega)})dt\le$$
        $$\le\parallel v_i\parallel_{L_2(0, T; L_4(\Omega))}\parallel v_j-u_j\parallel_{L_2(0, T; L_4(\Omega))}+$$
        $$+\parallel v_i-u_i\parallel_{L_2(0, T; L_4(\Omega))}\parallel u_j\parallel_{L_2(0, T; L_4(\Omega))}$$
        Отсюда, если
        $\parallel v - u \parallel L_2(0, T; (L_4(\Omega))^n) \rightarrow 0$,
        то $\parallel \phi_{ij}(u) - \phi_{ij}(v) \parallel L_1(0, T; L_2(\Omega)) \rightarrow 0$.
        Поэтому каждое из отображений $\phi_{ij}$, а следовательно, и отображение $K$ непрерывны.
    \end{enumerate}
\end{proof}

По утверждению леммы $\Delta v \in L_2(0, T; V^\ast), K(v) \in L_1(0, T; VV^\ast)$, поэтому $\Delta v(t) + K(v(t)) + f(t) \in L_1(0, T; VV^\ast)$.
Тогда из равенства (\ref{eq:5.8}) и теоремы 4.6 следует
\begin{enumerate}
    \item что функция $v(t)$ имеет суммируемую производную $v_\prime(t)$;
    \item в силу равенства (4.3)
    $$\frac{d}{dt} \langle v(t), \ u \rangle = \langle v'(t), \ u \rangle $$
    \item равенство (5.8) можно записать в виде
    $$v^\prime(t)=\Delta v(t)+K(v(t))+f(t)$$
\end{enumerate}

Подводя итог рассуждений, отметим, что так как $v^\prime(t) \in L_1(0, T; V^\ast)$, то $v \in W_{2, 1} c X_0 = V, \ X_1 = V^\ast$. Поэтому в силу леммы 4.5 функция
$v(t)$ непрерывна на отрезке $[0, T]$ со значениями в $V^\ast$. Кроме того, по лемме 4.6 эта функция слабо непрерывна со значениями в $H$. Поэтому
начальное условие (\ref{eq:5.7}) имеет смысл.

Переформулируем понятие слабого решения
\begin{definition}
    Пусть $n \le 4, \ f \in L_2(0, T; (L_2(\Omega))^n)$ и $v^0 \in H$. Слабым (вариационным) решением задачи (\ref{eq:5.1})-(\ref{eq:5.4}) называется функция
    $v \in L_2(0, T; V)$ такая, что $v^0 \in L_1(0, T; V^\ast)$, равенство
    \begin{equation}\label{eq:5.11}
        \begin{gathered}
            v^\prime(t) - \Delta v(t) - K(v(t)) = f(t)
        \end{gathered}
    \end{equation}
    выполняется для почти всех значений $t \in (0, T)$ и
    \begin{equation}\label{eq:5.12}
        \begin{gathered}
            v(0) = v^0
        \end{gathered}
    \end{equation}
\end{definition}


\subsection {О единственности слабого и полного слабого решений в случае $n = 2$}
В этом разделе мы рассмотрим вопрос о единственности слабого и полного слабого решений краевой задачи (\ref{eq:5.1})-(\ref{eq:5.4}) для эволюционной
системы уравнений Навье"=Стокса. Будем показано, что в случае $\Omega \subset R^2$ слабое и полное слабое решение краевой задачи единственно.
Однако для размерности $n > 2$ аналогичное утверждение неверно. Примером, показывающим, что слабое решение задачи не единственно, служит результат
о бифуркации решений, содержащийся, например, в [22, гл.II, \S 4,п.4.4].

Сформулируем и докажем утверждение о единственности слабого решения в случае $n = 2$, следуя [22, гл.III, \S 3, теорема 3.2].

\begin{theorem}
    Пусть $\Omega$ ограниченная область в $R^2$ с локально липшицевой границей. Тогда слабое решение $v$ и полное слабое решение $(v, p)$
    (при условии $(p) \Omega = 0$) задачи (\ref{eq:5.1})-(\ref{eq:5.4}) единственно. Кроме того, функция $v$ непрерывна на отрезке $[0, T]$ со значениями в $H$ и
    \begin{equation}\label{eq:5.35}
        \begin{gathered}
            v(t) \rightarrow v_0 \ \textrm{в} \ H \ \textrm{при} \ t \rightarrow \infty.
        \end{gathered}
    \end{equation}
\end{theorem}
\begin{proof}
    Достаточно установить единственность слабого решения $v$, так как компонента $p$ полного слабого решения определяется компонентой $v$
    из равенства (5.34) единственным образом.

    Пусть v -- решение задачи (5.31), (5.32). Покажем, что $v \in W$, т.е. $v^\prime \in L_2(0, T; V^\ast)$.
    Воспользуемся оценкой
    $$\parallel K_\varepsilon (v) \parallel_{(H^{-1}(\Omega))^n} \le c_0 \parallel v \parallel^2_{(L_4(\Omega))^n},$$
    полученной при выводе неравенства (3.10), в случае $n = 2$ и $\varepsilon = 0$. Применяя неравенство О.А. Ладыженской (1.7), получим для любого $t \in [0, T]$
    $$\sup\limits_{v\in V}\langle K(v), \ \frac{u}{\parallel v\parallel}
    \rangle\le c_02^{1/2}(\int\limits_\Omega v(t)^2dt)^{1/2}\cdot\int\limits_\Omega\sum_{i=1}^{n=3}\frac{\partial v^2}{\partial x_i}dt$$
    Отсюда, возводя обе части неравенства в квадрат и интегрируя по $t$ на отрезке $[0, T]$, приходим к оценке
    $$\int_0^T \sup(\langle K(v), \ \frac{u}{\parallel v\parallel}\rangle)^2dt \le 2c_0\int_0^T(\int\limits_\Omega v(t)^2)
    \cdot\int\limits_\Omega\sum_{i=1}^{n=3}\frac{\partial v^2}{\partial x_i}dt\le$$
    $$\le 2c_0 \max\limits_{t\in[0, T]}\int\limits_\Omega v(t)^2dt\cdot\int_0^T\int\limits_\Omega\sum_{i=1}^{n=3}\frac{\partial v^2}{\partial x_i}dt$$
    Следовательно,
    $$(\int_0^T \sup(\langle K(v), \ \frac{u}{\parallel v\parallel}\rangle)^2dt)^{1/2}\le c_0 2^{1/2}\max\limits_{t\in[0, T]}
    \int\limits_\Omega v(t)^2dt\cdot\int_0^T v(t)^2dt$$
    Из представления $v^\prime = \Delta v(t) + K(v(t)) + f(t)$ очевидно, что $v^\prime \in L_2(0, T; V^\ast)$ и $v \in W$.
    Воспользовавшись вложением $W \subset C([0, T], H)$, получаем $v \in C([0, T], H)$ и заключение (\ref{eq:5.35}) теоремы.

    Покажем теперь единственность слабого решения. Предположим, что $u$ и $v$ -- слабые решения задачи (5.31), (5.32).
    Подставим эти решения в уравнение (5.31) и рассмотрим разность полученных равенств. Для разности $w = v - u$ получим равенство
    $$w^\prime - \Delta w(t) - K(v(t)) + K(u(t)) = 0$$
    Применим функционалы, стоящие в равенстве, к функции $w(t)$
    \begin{equation}\label{eq:5.36}
        \begin{gathered}
            \frac{1}{2}\frac{d}{dt}\int\limits_\Omega w(t, x)w(t, x)dt+v(w(t, x),w(t, x))= \\
            =\int\limits_\Omega\frac{\partial v_i}{\partial t}\cdot v(t, x)\frac{\partial w(t, x)}{\partial x_i}dx-\int\limits_\Omega\frac{\partial u_i}{\partial t}v(t, x)\cdot\frac{\partial w(t, x)}{\partial x_i}dx
        \end{gathered}
    \end{equation}
    Оценим правую часть полученного равенства.
    $$\int\limits_\Omega\frac{\partial v_i}{\partial t}v(t, x)\frac{\partial w(t, x)}
    {\partial x_i}dx-\int\limits_\Omega\frac{\partial u_i}{\partial t}v(t, x)\frac{\partial w(t, x)}{\partial x_i}dx=$$
    $$=\int\limits_\Omega\frac{\partial v_i}{\partial t}v(t, x)\frac{\partial w(t, x)}
    {\partial x_i}dx-\int\limits_\Omega\frac{\partial u_i}{\partial t}v(t, x)\frac{\partial w(t, x)}{\partial x_i}dx+$$
    $$+\int\limits_\Omega\frac{\partial u_i}{\partial t}v(t, x)\frac{\partial w(t, x)}
    {\partial x_i}dx-\int\limits_\Omega\frac{\partial u_i}{\partial t}v(t, x)\frac{\partial w(t, x)}{\partial x_i}dx=$$
    $$=\int\limits_\Omega\frac{\partial w_i(t, x)}{\partial t}\cdot v(t, x)\frac{\partial w(t, x)}
    {\partial x_i}-\int\limits_\Omega\frac{\partial u_i}{\partial t}w(t, x)\cdot\frac{\partial w_i(t, x)}{\partial x_i}dx$$
    Используем интегрирование по частям для вычисления первого интеграла
    $$\int\limits_\Omega\frac{\partial w_i(t, x)}{\partial t}v(t, x)\cdot w(t, x)=
    -\int\limits_\Omega\frac{\partial w_i(t, x)}{\partial t}\cdot\frac{\partial v(t, x)}{\partial x_i}w(t, x)dx$$
    так как $\partial_iw_i(t, x) = \textrm{div } w(t, x)=0$. Используем интегрирование по частям для вычисления второго из интегралов
    $$\int\limits_\Omega\frac{\partial u_i}{\partial t}w(t, x)\frac{\partial w(t, x)}{\partial x_i}dx=
    \sum_{i=0}^{n=2}\int\limits_\Omega\frac{\partial u_i}{\partial t}\cdot\frac{1}{2}\cdot\frac{\partial \mid w\mid^2(t, x)}{\partial x_i}dx=$$
    $$=-\frac{1}{2}\int\limits_\Omega\sum_{i=1}^{n=2}\frac{\partial^2 u_i(t, x)}{\partial t\partial x_i}\cdot\mid w\mid^2(t, x)dx=0$$
    так как $\sum\limits_{i=1}^{2}\partial_iu_i(t, x)= \textrm{div } u(t, x)=0$
    Отсюда и из равенства (\ref{eq:5.36}) получим
    $$\frac{1}{2}\frac{d}{dt}\int\limits_\Omega w(t, x)w(t, x)dx+v(w(t, x)w(t, x))=$$
    $$=-\int\limits_\Omega\frac{\partial w_i(t, x)}{\partial t}\cdot\frac{\partial v(t, x)}{\partial x_i}w(t, x)dx$$
    или
    \begin{equation}\label{eq:5.37}
        \begin{gathered}
            \frac{1}{2}\frac{d}{dt}\parallel w(t)\parallel_H^2+v\parallel w(t)\parallel_H^2=
            \bigg|\int\limits_\Omega w_i(t, x)\cdot\partial_iv(t, x)w(t, x)dx\bigg|
        \end{gathered}
    \end{equation}
    Оценим правую часть неравенства, используя неравенства Шварца,
    $$\bigg|\int\limits_\Omega w_i(t, x)\cdot\partial_iv(t, x)w(t, x)dx\bigg|=
    \bigg|\int\limits_\Omega w_i(t, x)\cdot\partial_j(t, x)w_j(t, x)dx\bigg|\le$$
    $$\left(\int\limits_\Omega \mid w_i(t, x)\mid^2|w_j(t, x)\mid^2dx \right)^{1/2}
    \cdot\left(\int\limits_\Omega\mid\partial_iv_j(t, x) \mid^2dx \right)^{1/2}\le$$
    $$\left(\int\limits_\Omega \mid w_i(t, x)\mid^4dx\right)^{1/4}\left(\int\limits_\Omega \mid w_j(t, x)\mid^4dx\right)^{1/4}
    \cdot\left(\int\limits_\Omega\mid\partial_iv_j(t, x) \mid^2dx \right)^{1/2}$$
    или
    $$\bigg|\int\limits_\Omega w_i(t, x)\cdot\partial_iv(t, x)w(t, x)dx\bigg|\le$$
    $$\le\parallel w_i(t)\parallel_{L_4(\Omega)}\cdot\parallel w_j(t)\parallel_{L_4(\Omega)}\cdot\parallel\partial_iv_j(t)\parallel_{L_2(\Omega)}$$
    Учитывая, что запись, содержащая повторяющиеся индексы, предполагает суммирование по этим индексам, получаем
    $$\bigg|\int\limits_\Omega w_i(t, x)\cdot\partial_iv(t, x)w(t, x)dx\bigg|\le\parallel w(t)\parallel_{(L_4(\Omega))^n}^2\cdot\parallel v(t)\parallel_V$$
    Применим неравенство О.А. Ладыженской и далее неравенство Коши \linebreak
    $a\cdot b=\varepsilon a^2+\frac{b^2}{4\varepsilon}c\varepsilon=\frac{v}{2^{1/2}}$, получим
    $$\bigg|\int\limits_\Omega w_i(t, x)\cdot\partial_iv(t, x)w(t, x)dx\bigg|\le$$
    $$\le 2^{1/2}\parallel w(t)\parallel_{(L_2(\Omega))^n}\cdot\parallel w(t)\parallel_V\cdot\parallel v(t)\parallel_V\le$$
    $$\le v\parallel w(t)\parallel_V^2+\frac{1}{2^{3/2}v}\parallel v(t)\parallel_{(L_2(\Omega))^n}^2\cdot\parallel w(t)\parallel_V$$
    Подставляя полученное соотношение в неравенство (\ref{eq:5.37}), получаем
    $$\frac{1}{2}\frac{d}{dt}\parallel w(t)\parallel_H^2\le\frac{1}{2^{3/2}v}\parallel w(t)\parallel_H^2\cdot\parallel v(t)\parallel_V$$
    Тогда из неравенство Гронуолла"=Беллмана [1, теорема 2б глава IV, с.188] следует
    $$\parallel w(t)\parallel_H^2\le\parallel w(0)\parallel_H^2\exp\left(\int_0^t\frac{1}{2^{1/2}v}\cdot\parallel v(s)\parallel_Vds\right)$$
    Поскольку $w(0) = v(0)- u(0) = 0$, то из полученного выше неравенства приходим к выводу,
    что $w(t)=0$ для всех $t\in[0, T]$. Следовательно, $v=u$ и слабое и полное слабое решение задачи (\ref{eq:5.1})-(\ref{eq:5.4}) единственно.
\end{proof}
