\documentclass[10pt, pdf, hyperref={unicode}]{beamer}
\usepackage[T2A]{fontenc}
\usepackage[utf8]{inputenc}
\usepackage[english, russian]{babel}
\usepackage{amssymb, amsfonts, amsmath, amsthm, microtype, pdfpages}

\usetheme{Madrid}
\usecolortheme{beaver}

\title{<<Изучение единственности слабых решений системы Навье-Стокса>>}
\date{22.06.2020}
\author{Мукасеева Дарья Александровна}

\setbeamertemplate{frametitle}[default][center]
\setbeamertemplate{navigation symbols}{}
\setbeamertemplate{footline}[page number]
\setbeamertemplate{caption}[numbered]

\newtheorem{ru_theo}{Теорема}
\renewenvironment{theorem}{\begin{ru_theo}}{\end{ru_theo}}
\newtheorem{ru_def}{Определение}
\renewenvironment{definition}{\begin{ru_def}}{\end{ru_def}}
\newtheorem{ru_lemma}{Лемма}
\renewenvironment{lemma}{\begin{ru_lemma}}{\end{ru_lemma}}


\begin{document}

  \begin{frame} % титульный лист 
    \titlepage
    \begin{center}
      Бакалаврская работа\\
      Направление 01.03.01 Математика\\
      Профиль Математическое моделирования\\
      Руководитель: Звягин Андрей Викторович
    \end{center}
  \end{frame}


  \begin{frame}
    \frametitle{Понятие слабого решения}
    \begin{center}
      \begin{minipage}[h]{0.97\linewidth}
        Пусть $\Omega$ --- ограниченная область в пространстве $R^n$, где $n=2, 3$,
        с достаточно гладкой границей $\partial\Omega$.
        Рассмотрим начально"=краевую задачу для системы уравнений Навье"=Стокса
        \begin{equation}\label{eq:5.1}
          \begin{gathered}
            \frac{\partial v}{\partial t}+\sum_{i=1}^nv_i\frac{\partial v}
            {\partial x_i}-\nu\Delta v+\nabla p=f;
          \end{gathered}
        \end{equation}

        \begin{equation}\label{eq:5.2}
          \begin{gathered}
            \operatorname{div} v=0;
          \end{gathered}
        \end{equation}

        \begin{equation}\label{eq:5.3}
          \begin{gathered}
            v|_{t=0}=v_0;
          \end{gathered}
        \end{equation}
        \begin{equation}\label{eq:5.4}
          \begin{gathered}
            v|_{(0, T)\times\partial\Omega}=0.
          \end{gathered}
        \end{equation}
      \end{minipage}
    \end{center}
  \end{frame}


  \begin{frame}
    \frametitle{Определение сильного решения}
    \begin{center}
      \begin{minipage}[h]{0.97\linewidth}
        \begin{definition}
          Сильным решением начально"=краевой задачи (\ref{eq:5.1})-(\ref{eq:5.4})
          называется пара функций $v\in L_2(0, T; L_2(\Omega))$ и
          $p\in L_2(0, T; L_2(\Omega))$, удовлетворяющих следующим условиям:
        \end{definition}
        \begin{enumerate} 
          \item обобщенные частные производные функций, содержащихся в равенствах
          (\ref{eq:5.1})-(\ref{eq:5.4}), принадлежат пространству $L_2(0, T; L_2(\Omega))$;
          \item при подстановке функций уравнения (\ref{eq:5.1})-(\ref{eq:5.2})
          обращаются в равенства в пространстве $L_2(0, T; L_2(\Omega))$;
          \item функция $v$ удовлетворяет начальному условию (\ref{eq:5.3}) и граничному условию (\ref{eq:5.4}).
        \end{enumerate}
      \end{minipage}
    \end{center}
  \end{frame}


  \begin{frame}
    \frametitle{Определение слабого решения}
    \begin{center}
      \begin{minipage}[h]{0.97\linewidth}
\begin{definition}
    Пусть $f\in L_2(0,T;L_2(\Omega))$ и $v_0\in H$. Слабым решением задачи (\ref{eq:5.1})-(\ref{eq:5.4}) называется функция $v\in L_2(0, T;V)$,
    удовлетворяющая для всех $\varphi \in V$ и для почти всех значений $t \in (0, T)$ равенству
    \begin{equation}\label{eq:5.6}
        \begin{gathered}
            \frac{d}{dt}\int\limits_\Omega v \varphi dx-\sum_{i=1}^{n}\int\limits_\Omega v_i v
            \frac{\partial \varphi }{\partial x_i}dx + \nu\int\limits_\Omega \nabla v :\nabla\varphi dx=
            \int\limits_\Omega f \varphi dx
        \end{gathered}
    \end{equation}
    и условию
    \begin{equation}\label{eq:5.7}
        \begin{gathered}
            v(0) = v_0.
        \end{gathered}
    \end{equation}
\end{definition}
      \end{minipage}
    \end{center}
  \end{frame}


  \begin{frame}
    \frametitle{Доказательство линейности и непрерывности оператора $\Delta:L_2 (0,T;V)\rightarrow L_2 (0,T;V^*)$}
    \begin{center}
      \begin{minipage}[h]{0.97\linewidth}
      \begin{lemma} ~\\
    \begin{enumerate}
        \item Оператор $\Delta: L_2(0, T; V) \rightarrow L_2(0, T; V^\ast)$ линейный и\linebreak
        непрерывный, причем
        \begin{equation}\label{eq:5.9}
            \begin{gathered}
                \| \Delta v \|_{L_2(0, T; V^\ast)} = \| v \|_{L_2(0, T; V)}, \ \forall v \in L_2(0, T; V^\ast).
            \end{gathered}
        \end{equation}
        \item Оператор $K: L_2(0, T; V) \rightarrow L_1(0, T; V^\ast)$ непрерывен и справедлива оценка
        \begin{equation}\label{eq:5.10}
            \begin{gathered}
                \| K(v) \|_{L_1(0, T; V^\ast)} \leqslant C_2\| v \|^2_{L_2(0, T;V)}, \ \forall v \in L_2(0, T; V^\ast),
            \end{gathered}
        \end{equation}
    \end{enumerate}
    для некоторой константы $C_2$.
\end{lemma}
$$\langle K(v),\varphi \rangle = \sum_{i=1}^{n}\int\limits_\Omega v_i v \frac{\partial \varphi}{\partial x_i}dx.$$
      \end{minipage}
    \end{center}
  \end{frame}


\begin{frame}
    \frametitle{Полученные результаты}
    \begin{center}
      \begin{minipage}[h]{0.97\linewidth}
        \begin{lemma}
          Для $p_0\ge 1, \ p_1\ge 1$ имеет место вложение\linebreak
          $W_{p_0,p_1}=\{v\in L_{p_0}(a,b,X_0), \ v^{\prime}\in L_{p_0}(a,b,X_0)\}\subset C([a,b],X_1)$
          и это вложение непрерывно.
        \end{lemma}
        \begin{lemma}
          Пусть $X$ и $Y$ --- банаховы пространства, такие, что $X$ --- рефлексивно и вложение $X\subset Y$ непрерывно.
          Если функция $v\in L_{\infty}(a,b;X)$ слабо непрерывна как функция со значениями в $Y$,то и слабо непрерывна и как функция со значениями в $X$.
        \end{lemma}
      \end{minipage}
    \end{center}
  \end{frame}

  \begin{frame}
    \frametitle{Определение слабого решения}
    \begin{center}
      \begin{minipage}[h]{0.97\linewidth}
        \begin{definition}
          Пусть $n \le 4, \ f \in L_2(0, T; L_2(\Omega))$ и $v \in H$. Слабым решением задачи (\ref{eq:5.1})-(\ref{eq:5.4}) называется функция
          $v \in L_2(0, T; V)\cap L_{\infty}(0,T;H)$ и условию $ v^{\prime}\in L_1(0, T; V^\ast)$,
          удовлетворяющая при почти всех значений $t \in (0, T)$ равенству 
          \begin{equation}\label{eq:5.11}
              \begin{gathered}
                v^\prime(t) - \nu\Delta v(t) - K(v(t)) = f(t)
              \end{gathered}
          \end{equation}
          и начальному условию
          \begin{equation}\label{eq:5.12}
              \begin{gathered}
                v(0) = v_0.
              \end{gathered}
          \end{equation}
        \end{definition}
        Для системы уравнений Навье"=Стокса (\ref{eq:5.1})-(\ref{eq:5.4}) Ж. Лере в 1934 был получен следующий результат:
        \begin{theorem}
          Пусть $n=2,3$. Для каждой функции $f\in L_2(0,T;V^\ast)$ и $v_0\in H$ начально-краевая задача (\ref{eq:5.1})-(\ref{eq:5.4})
          имеет хотя бы одно слабое решение $v.$
        \end{theorem}
      \end{minipage}
    \end{center}
  \end{frame}


    \begin{frame}
    \frametitle{Единственность слабого решения}
    \begin{center}
      \begin{minipage}[h]{0.97\linewidth}
        \begin{theorem}
          Пусть $\Omega$ ограниченная область в $R^2$ с достаточно гладкой границей $\partial\Omega$. Тогда слабое решение $v$  решение
          задачи (\ref{eq:5.1})-(\ref{eq:5.4}) единственно.
        \end{theorem}
        \begin{equation}\label{eq:5.11}
        \begin{gathered}
            v^\prime(t) - \nu\Delta v(t) - K(v(t)) = f(t)
        \end{gathered}
    \end{equation}
      \end{minipage}
    \end{center}
  \end{frame}

  \begin{frame}
    \frametitle{Оценка из доказательства}
    \begin{center}
      \begin{minipage}[h]{0.97\linewidth}
    \begin{equation}\label{eq:5.12}
      \begin{gathered}
        $$w^\prime - \nu\Delta w(t) - K(v(t)) + K(u(t)) = 0$$
      \end{gathered}
    \end{equation}
      \begin{equation}\label{eq:5.37}
        \begin{gathered}
            \frac{1}{2}\frac{d}{dt}\| w(t)\|_H^2+\nu\| w(t)\|_V^2\le
            \bigg|\sum_{i=1}^{n=2}\int\limits_\Omega w_i(t, x)\frac{\partial v(t, x)}{\partial x_i}w(t, x)dx\bigg|.
        \end{gathered}
    \end{equation}
    \begin{equation}\label{eq:5.38}
        \begin{gathered}
        $$\| w(t)\|_H^2\le\| w(0)\|_H^2\exp\left(\int_0^t\frac{1}{2^{1/2}\nu}\| v(s)\|_Vds\right).$$
        \end{gathered}
    \end{equation}
      \end{minipage}
    \end{center}
  \end{frame}

  \begin{frame}
    \begin{alertblock}{}
      \centerline{\large Спасибо за внимание!}
    \end{alertblock}
  \end{frame}
\end{document}
