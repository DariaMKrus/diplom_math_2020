\documentclass[10pt, pdf, hyperref={unicode}]{beamer}
\usepackage[T2A]{fontenc}
\usepackage[utf8]{inputenc}
\usepackage[english, russian]{babel}
\usepackage{amssymb, amsfonts, amsmath, amsthm, microtype, pdfpages}

\usetheme{Madrid}
\usecolortheme{beaver}

\title{<<Изучение единственности слабых решений системы Навье-Стокса>>}
\date{22.06.2020}
\author{Мукасеева Дарья Александровна}

\setbeamertemplate{frametitle}[default][center]
\setbeamertemplate{navigation symbols}{}
\setbeamertemplate{footline}[page number]
\setbeamertemplate{caption}[numbered]

\newtheorem{ru_theo}{Теорема}
\renewenvironment{theorem}{\begin{ru_theo}}{\end{ru_theo}}
\newtheorem{ru_def}{Определение}
\renewenvironment{definition}{\begin{ru_def}}{\end{ru_def}}
\newtheorem{ru_lemma}{Лемма}
\renewenvironment{lemma}{\begin{ru_lemma}}{\end{ru_lemma}}

\begin{document}

  \begin{frame} % титульный лист 
    \titlepage
    \begin{center}
      Бакалаврская работа\\
      Направление 01.03.01 Математика\\
      Профиль Математическое моделирования
    \end{center}
  \end{frame}


  \begin{frame}
    \frametitle{Понятие слабого решения}
    \begin{center}
      \begin{minipage}[h]{0.97\linewidth}
        Пусть $\Omega$ --- ограниченная область в пространстве $R^n$, где $n=2, 3$,
        с достаточно гладкой границей $\partial\Omega$.
        Рассмотрим начально"=краевую задачу для системы уравнений Навье"=Стокса
        \begin{equation}\label{eq:5.1}
          \begin{gathered}
            \frac{\partial v}{\partial t}+\sum_{i=1}^nv_i\frac{\partial v}
            {\partial x_i}-\nu\Delta v+\nabla p=f;
          \end{gathered}
        \end{equation}

        \begin{equation}\label{eq:5.2}
          \begin{gathered}
            \operatorname{div} v=0;
          \end{gathered}
        \end{equation}

        \begin{equation}\label{eq:5.3}
          \begin{gathered}
            v|_{t=0}=v_0;
          \end{gathered}
        \end{equation}
        \begin{equation}\label{eq:5.4}
          \begin{gathered}
            v|_{(0, T)\times\partial\Omega}=0.
          \end{gathered}
        \end{equation}
      \end{minipage}
    \end{center}
  \end{frame}


  \begin{frame}
    \frametitle{Введение необходимых функциональных пространств}
    \begin{center}
      \begin{minipage}[h]{0.97\linewidth}
        $L_p(\Omega)$ --- множество измеримых функций, суммируемых с $p$-ой степенью, где $1\le p< \infty$,
        и нормой $\|v\|_{L_p(\Omega)}=(\int\limits_\Omega |v(x)|^pdx)^{1/p}$.

        Пространство $L_{\infty}(\Omega)$ состоит из измеримых существенно ограниченных функций $v:\Omega\rightarrow R^n$. 
        Функция $v:\Omega\rightarrow R^n$ называется существенной ограниченной,
        если существует число $C_1<\infty$, что $|v(x)|\le C_1$ при почти всех $x\in\Omega$.
        Норма в $L_{\infty}(\Omega)$ задается
        $\|v\|_{L_{\infty}(\Omega)}=ess \ \underset{x\in\Omega}{sup} |v(x)|$.

        $W_p^m(\Omega)$ --- где $m\geqslant 1, \ p\geqslant 1$, пространство Соболева, состоящее из
        функции, которые со своими обобщенными частными производными до порядка
        $m$ включительно принадлежат пространству $L_p(\Omega)$.\\
        Норма в $W_p^m(\Omega)$ задается
        $\|v\|_{W_p^m(\Omega)}=\left(\sum\limits_{|\alpha|\leqslant m} \int\limits_\Omega |D^{\alpha}v(x)|^pdx\right)^{1/p}$.
      \end{minipage}
    \end{center}
  \end{frame}


  \begin{frame}
    \frametitle{Введение необходимых функциональных пространств}
    \begin{center}
      \begin{minipage}[h]{0.97\linewidth}
        $L_p(a,b;X)$ --- где $1\le p< \infty$ пространство суммируемых с
        $p$-ой степенью функций на $[a, b]$ со значениями в
        банаховом пространстве $X$. Норма пространства $L_p(a, b;X)$ задается 
        $\|v\|_{L_p(a, b;X)}=(\int\limits_0^T\|v(s)\|_X^p ds)^{1/p}.$

        Через $L_{\infty}(a,b;X)$ будем обозначать множество всех измеримых существенно ограниченных функций
        $v:[a,b]\rightarrow X$.
        Множество $L_{\infty}(a,b;X)$ является банаховым пространством относительно нормы
        $\|v\|_{L_{\infty}(a,b;X)}=ess \ \underset{x\in\Omega}{sup} \|v(s)\|_X$.
      \end{minipage}
    \end{center}
  \end{frame}


  \begin{frame}
    \frametitle{Определение сильного решения}
    \begin{center}
      \begin{minipage}[h]{0.97\linewidth}
        \begin{definition}
          Сильным решением начально"=краевой задачи (\ref{eq:5.1})-(\ref{eq:5.4})
          называется пара функций $v\in L_2(0, T; L_2(\Omega))$ и
          $p\in L_2(0, T; L_2(\Omega))$, удовлетворяющих следующим условиям:
        \end{definition}
        \begin{enumerate} 
          \item обобщенные частные производные функций, содержащихся в равенствах
          (\ref{eq:5.1})-(\ref{eq:5.4}), принадлежат пространству $L_2(0, T; L_2(\Omega))$;
          \item при подстановке функций уравнения (\ref{eq:5.1})-(\ref{eq:5.2})
          обращаются в равенства в пространстве $L_2(0, T; L_2(\Omega))$;
          \item функция $v$ удовлетворяет начальному условию (\ref{eq:5.3}) и граничному условию (\ref{eq:5.4}).
        \end{enumerate}
      \end{minipage}
    \end{center}
  \end{frame}

  \begin{frame}
    \frametitle{Определение слабого решения}
    \begin{center}
      \begin{minipage}[h]{0.97\linewidth}
\begin{definition}
    Пусть $f\in L_2(0,T;L_2(\Omega))$ и $v_0\in H$. Слабым решением задачи (\ref{eq:5.1})-(\ref{eq:5.4}) называется функция $v\in L_2(0, T;V)$,
    удовлетворяющая для всех $\varphi \in V$ и для почти всех значений $t \in (0, T)$ равенству
    \begin{equation}\label{eq:5.6}
        \begin{gathered}
            \frac{d}{dt}\int\limits_\Omega v \varphi dx-\sum_{i=1}^{n}\int\limits_\Omega v_i v
            \frac{\partial \varphi }{\partial x_i}dx + \nu\int\limits_\Omega \nabla v :\nabla\varphi dx=
            \int\limits_\Omega f \varphi dx
        \end{gathered}
    \end{equation}
    и условию
    \begin{equation}\label{eq:5.7}
        \begin{gathered}
            v(0) = v_0.
        \end{gathered}
    \end{equation}
\end{definition}
      \end{minipage}
    \end{center}
  \end{frame}

  \begin{frame}
    \frametitle{Доказательство линейности и непрерывности оператора $\Delta:L_2 (0,T;V)→L_2 (0,T;V^*)$}
    \begin{center}
      \begin{minipage}[h]{0.97\linewidth}
      \begin{lemma} ~\\
    \begin{enumerate}
        \item Оператор $\Delta: L_2(0, T; V) \rightarrow L_2(0, T; V^\ast)$ линейный и\linebreak
        непрерывный, причем
        \begin{equation}\label{eq:5.9}
            \begin{gathered}
                \| \Delta v \|_{L_2(0, T; V^\ast)} = \| v \|_{L_2(0, T; V)}, \ \forall v \in L_2(0, T; V^\ast).
            \end{gathered}
        \end{equation}
        \item Оператор $K: L_2(0, T; V) \rightarrow L_1(0, T; V^\ast)$ непрерывен и справедлива оценка
        \begin{equation}\label{eq:5.10}
            \begin{gathered}
                \| K(v) \|_{L_1(0, T; V^\ast)} \leqslant C_2\| v \|^2_{L_2(0, T;V)}, \ \forall v \in L_2(0, T; V^\ast),
            \end{gathered}
        \end{equation}
    \end{enumerate}
    для некоторой константы $C_2$.
\end{lemma}
      \end{minipage}
    \end{center}
  \end{frame}

    \begin{frame}
    \frametitle{Вывод из леммы}
    \begin{center}
      \begin{minipage}[h]{0.97\linewidth}
      По утверждению леммы $\nu\Delta v \in L_2(0, T; V^\ast), K(v) \in L_1(0, T; V^\ast)$, поэтому $\nu\Delta v(t) + K(v(t)) + f(t) \in L_1(0, T; V^\ast)$.
      \begin{enumerate}
      \item что функция $v(t)$ имеет суммируемую производную $v^\prime(t);$
      \item в силу равенства $\frac{d}{dt}\langle \phi, u(t) \rangle = \langle \phi, g(t) \rangle;$
      $$\frac{d}{dt} \langle v(t), \varphi \rangle = \langle v'(t), \varphi \rangle;$$
      \item равенство $\frac{d}{dt}\langle v(t), \varphi \rangle=\langle \Delta v(t)+K(v(t))+f(t),\varphi\rangle$ можно записать в виде
      $$v^\prime(t)=\nu\Delta v(t)+K(v(t))+f(t).$$
      \end{enumerate}
      \end{minipage}
    \end{center}
  \end{frame}

    \begin{frame}
    \frametitle{Принадлежность $v^{\prime} (t)$к пространству $ L_1 (0,T;V^*)$}
    \begin{center}
      \begin{minipage}[h]{0.97\linewidth}
        \begin{lemma}
          Для $p_0\ge 1, \ p_1\ge 1$ имеет место вложение\linebreak
          $W_{p_0,p_1}=\{v\in L_{p_0}(a,b,X_0), \ v_1\in L_{p_0}(a,b,X_0)\}\subset C([a,b],X_1)$
          и это вложение непрерывно.
        \end{lemma}
        \begin{lemma}
          Пусть $X$ и $Y$ --- банаховы пространства, такие, что $X$ --- рефлективно и вложение $X\subset Y$ непрерывно.
          Если функция $v\in L_{\infty}(a,b;X)$ слабо непрерывна как функция со значениями в $Y$,то и слабо непрерывна и как функция со значениями в $X$.
        \end{lemma}
      \end{minipage}
    \end{center}
  \end{frame}

    \begin{frame}
    \frametitle{Определение слабого решения}
    \begin{center}
      \begin{minipage}[h]{0.97\linewidth}
        \begin{definition}
    Пусть $n \le 4, \ f \in L_2(0, T; L_2(\Omega))$ и $v \in H$. Слабым решением задачи (\ref{eq:5.1})-(\ref{eq:5.4}) называется функция
    $v \in L_2(0, T; V)\cap L_{\infty}(0,T;H)$ и условию$ v^{\prime}\in L_1(0, T; V^\ast)$, удовлетворяющая при почти всех значений $t \in (0, T)$ равенству 
    \begin{equation}\label{eq:5.11}
        \begin{gathered}
            v^\prime(t) - \nu\Delta v(t) - K(v(t)) = f(t)
        \end{gathered}
    \end{equation}
      и начальному условию
    \begin{equation}\label{eq:5.12}
        \begin{gathered}
            v(0) = v_0.
        \end{gathered}
    \end{equation}
\end{definition}
Для системы уравнений Навье"=Стокса (\ref{eq:5.1})-(\ref{eq:5.4}) Ж. Лере в 1934 был получен следующий результат:
\begin{theorem}
Пусть $n=2,3$. Для каждой функции $f\in L_2(0,T;V^\ast)$ и $v_0\in H$ начально-краевая задача (\ref{eq:5.1})-(\ref{eq:5.4})
имеет хотя бы одно слабое решение $v.$
\end{theorem}
      \end{minipage}
    \end{center}
  \end{frame}

    \begin{frame}
    \frametitle{Единственность слабого решения}
    \begin{center}
      \begin{minipage}[h]{0.97\linewidth}
        \begin{theorem}
          Пусть $\Omega$ ограниченная область в $R^2$ с достаточно гладкой границей $\partial\Omega$. Тогда слабое решение $v$  решение
          задачи (\ref{eq:5.1})-(\ref{eq:5.4}) единственно.
        \end{theorem}
      \end{minipage}
    \end{center}
  \end{frame}


  % \begin{frame}
  %     \frametitle{Постановка задачи}
  %     \begin{center}
  %         \begin{minipage}[h]{0.97\linewidth}
  %             Для решения такой задачи необходимо на основе теории вибрационных машин и теоремы об оптимальности импульса Максвелла-Фейера
  %             разработать ПО для автоматизированного расчета характеристик импульсного погружателя
  %             с возможностью ввода начальных данных и наглядного вывода результатов\footnotemark[1]\footnotemark[2].
  %         \end{minipage}
  %     \end{center}
  %     \footnotetext[1]{\label{foot-1} Блехман И. И. Вибрационная механика. --- М. : Физико-математическая литература, 1994.}
  %     \footnotetext[2]{\label{foot-2} Костин Д. В. Бифуркация резонансных колебаний и оптимизация тригонометрического импульса по коэффициенту несимметрии // Математический сборник. --- М., 2016.}
  % \end{frame}


  % \begin{frame}
  %     \frametitle{Конструкция импульсного погружателя}
  %     \begin{center}
  %         \begin{minipage}[h]{0.97\linewidth}
  %             \begin{minipage}[h]{0.55\linewidth}
  %                 Работа погружателя основана на двух основных принципах:
  %                 \begin{enumerate} 
  %                     \item На эффекте резкого снижения сопротивлению погружения свайного элемента при сообщении последнему вибрации;
  %                     \item На действии полигармонического импульса, создаваемого центробежными силами системы дебалансов.
  %                 \end{enumerate}
  %                 При вращении валов (1) с дебалансами (2) на их ось крепления действует центробежная сила и погружатель получает вибрирующее движение,
  %                 которое через наголовник (3) сообщается свайному элементу (4).
  %             \end{minipage}
  %             % \hfill
  %             % \begin{minipage}[h]{0.36\linewidth}
  %             %     \begin{figure}[h]
  %             %         \centering
  %             %         \includegraphics[width=1\linewidth]{../img/scheme_porg_2.png}
  %             %         \caption{Схема импульсного погружателя.}
  %             %     \end{figure}
  %             % \end{minipage}
  %         \end{minipage}
  %     \end{center}
  % \end{frame}


  % \begin{frame}
  %     \frametitle{Конструкция дебаланса}
  %     \begin{center}
  %         \begin{minipage}[h]{0.97\linewidth}
  %             \begin{minipage}[h]{0.6\linewidth}
  %                 Пусть дан  дебаланс с радиусом $r$, радиус вала которого равен $R$,
  %                 $\omega$ --- угловая скорость и $l$ --- расстояние от центра масс до оси вращения дебаланса, а его масса будет равна $m$.
  %                 Центробежная сила:
  %                 \begin{equation}
  %                     \begin{gathered}
  %                         F_{\textrm{центр.}} = m \cdot \omega^2 \cdot l \\
  %                         \textrm{где } l = \frac{4 r}{3 \pi}
  %                     \end{gathered}
  %                 \end{equation}
  %                 \\
  %                 Гармонические колебания:
  %                 \begin{equation}
  %                     \begin{gathered}
  %                         x(t) = \lambda \cos (\omega t) \\
  %                         \textrm{где } \lambda = m \cdot \omega^2 \cdot l
  %                     \end{gathered}
  %                 \end{equation}
  %             \end{minipage}
  %             % \hfill
  %             % \begin{minipage}[h]{0.36\linewidth}
  %             %     \begin{figure}[h]
  %             %         \centering
  %             %         \includegraphics[width=1\linewidth]{../img/debalance.png}
  %             %         \caption{Схема дебаланса.}
  %             %     \end{figure}
  %             % \end{minipage}
  %         \end{minipage}
  %     \end{center}
  % \end{frame}


  % \begin{frame}
  %     \frametitle{Конструкция пары дебалансов}
  %     \begin{center}
  %         \begin{minipage}[h]{0.97\linewidth}
  %             Для компенсации горизонтальных сил в конструкции погружателя используются парные дебалансы.\\
  %             Гармонические колебания пары дебалансов:
  %             \begin{equation}
  %                 \begin{gathered}
  %                     x(t) = 2 \lambda \cos (\omega t), \textrm{где } \lambda = m \cdot \omega^2 \cdot l
  %                 \end{gathered}
  %             \end{equation}
  %             % \begin{figure}[h]
  %             %     \centering
  %             %     \includegraphics[width=0.52\linewidth]{../img/double_debalance.png}
  %             %     \caption{Схема пары дебалансов.}
  %             % \end{figure}
  %         \end{minipage}
  %     \end{center}
  % \end{frame}


  % \begin{frame}
  %     \frametitle{Гармонические колебания дебалансов}
  %     \begin{center}
  %         \begin{minipage}[h]{0.97\linewidth}
  %             При использовании нескольких пар дебалансов, вышестоящий уровень дебалансов должен иметь угловую скорость в два раза выше, чем прошлый.\\
  %             \newline
  %             \begin{minipage}[h]{0.44\linewidth}
  %                 % \begin{figure}[h]
  %                 %     \centering
  %                 %     $x(t) = \lambda_1 \cos (\omega t)$
  %                 %     \includegraphics[width=1\linewidth]{../grap/impulse_1.pdf}
  %                 %     \caption{Гармонические колебания для одной пары дебалансов.}
  %                 % \end{figure}
  %             \end{minipage}
  %             \hfill
  %             \begin{minipage}[h]{0.44\linewidth}
  %                 % \begin{figure}[h]
  %                 %     \centering
  %                 %     $x(t) = \lambda_1 \cos (\omega t) + \lambda_2 \cos (2\omega t)$
  %                 %     \includegraphics[width=1\linewidth]{../grap/impulse_2.pdf}
  %                 %     \caption{Гармонические колебания для двух пар дебалансов.}
  %                 % \end{figure}
  %             \end{minipage}
  %         \end{minipage}
  %     \end{center}
  % \end{frame}

  % \begin{frame}
  %     \frametitle{Гармонические нескольких пар дебалансов}
  %     \begin{center}
  %         \begin{minipage}[h]{0.97\linewidth}
  %             Гармоническое колебания для $n$ дебалансов, где $k$ --- порядковый номер пары дебалансов, будет иметь вид:
  %             \begin{equation}\label{eq:harmonic_sum}
  %                 \begin{gathered}
  %                     F = \sum\limits_{k = 1}^n 2 \lambda_k \cdot \cos (k \omega t), \lambda = m \cdot \omega^2 \cdot l
  %                 \end{gathered}
  %             \end{equation}
  %             Использование нескольких пар дебалансов разных характеристик позволяет увеличить импульс,
  %             направленный на погружение свайного элемента и уменьшить импульс, направленный в противоположную сторону.
  %         \end{minipage}
  %     \end{center}
  % \end{frame}


  % \begin{frame}
  %     \frametitle{Задача оптимизации}
  %     \begin{center}
  %         \begin{minipage}[h]{0.97\linewidth}
  %             Пусть $f_{\max}(t)$ --- максимальное значение импульса силы за время $t$, $f_{\min}(t)$ --- минимальное значение импульса за время $t$. Тогда:
  %             \begin{equation}
  %                 \begin{gathered}
  %                     K = \left| \frac{f_{\max}(t)}{f_{\min}(t)} \right| \rightarrow \max
  %                 \end{gathered}
  %             \end{equation}
  %             \begin{block}{Теорема\footnotemark[3]}\label{teorema}
  %                 Многочлен (\ref{eq:harmonic_sum}) является оптимальным т. и т. т. , когда он с точностью до постоянного множителя имеет вид суммы Фейера:
  %                 \begin{equation}\label{eq:feer}
  %                     \begin{gathered}
  %                         f_n(t) = \sum\limits_{k = 1}^n (n + 1 - k) \cos(kt)\\
  %                         \max \limits_{\lambda} K_n(\lambda) = n
  %                     \end{gathered}
  %                 \end{equation}
  %             \end{block}
  %         \end{minipage}
  %     \end{center}
  %     \footnotetext[3]{\label{foot-3} Костин Д. В. Бифуркация резонансных колебаний и оптимизация тригонометрического импульса по коэффициенту несимметрии
  %     // Математический сборник. --- М., 2016. — С. 90—109.}
  % \end{frame}

  % \begin{frame}
  %     \frametitle{Задача оптимизации}
  %     \begin{center}
  %         \begin{minipage}[h]{0.97\linewidth}
  %             Исходя из теоремы выше, следует, что:
  %             \begin{equation}
  %                 \begin{gathered}
  %                     \lambda_k = \frac{n - k + 1}{n} \cdot \lambda_1,\\
  %                     \textrm{где } \lambda_1 = m_1 \cdot \omega_{1}^{2} \cdot l_1
  %                 \end{gathered}
  %             \end{equation}
  %             Это позволяет найти коэффициент $\lambda_k$ для $k$-й пары дебалансов, когда общее количество дебалансов погружателя --- $n$.
  %         \end{minipage}
  %     \end{center}
  % \end{frame}

  % \begin{frame}
  %     \frametitle{Программная реализация}
  %     \begin{center}
  %         \begin{minipage}[h]{0.97\linewidth}
  %             При помощи применения теоремы об оптимальности модели полигармонического импульса и на основе теории вибрационных машин
  %             на языке Python была разработана программа для автоматического расчета характеристик дебалансов погружателя.
  %             % \begin{figure}[h]
  %             %     \centering
  %             %     \includegraphics[width=0.62\linewidth]{../img/xolm_2.png}
  %             %     \caption{Скриншот программы.}
  %             % \end{figure}
  %         \end{minipage}
  %     \end{center}
  % \end{frame}


  \begin{frame}
    \begin{alertblock}{}
      \centerline{\large Спасибо за внимание!}
    \end{alertblock}
  \end{frame}
\end{document}
