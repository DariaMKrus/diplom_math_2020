\documentclass[10pt, pdf, hyperref={unicode}]{beamer}
\usepackage[T2A]{fontenc}
\usepackage[utf8]{inputenc}
\usepackage[english, russian]{babel}
\usepackage{amssymb, amsfonts, amsmath, amsthm, microtype, pdfpages}

\usetheme{Madrid}
\usecolortheme{beaver}

\title{<<Изучение единственности слабых решений системы Навье-Стокса>>}
\date{22.06.2020}
\author{Мукасеева Дарья Александровна}

\setbeamertemplate{frametitle}[default][center]
\setbeamertemplate{navigation symbols}{}
\setbeamertemplate{footline}[page number]
\setbeamertemplate{caption}[numbered]


\begin{document}

    \begin{frame} % титульный лист 
        \titlepage
        \begin{center}
            Бакалаврская работа\\
            Направление 01.03.01 Математика\\
            Профиль Математическое моделирования
        \end{center}
    \end{frame}


    \begin{frame}
        \frametitle{Понятие слабого решения}
        \begin{center}
            \begin{minipage}[h]{0.97\linewidth}
Пусть $\Omega$ --- ограниченная область в пространстве $R^n$, где $n=2, 3$, с достаточно гладкой границей $\partial\Omega$.
Рассмотрим начально"=краевую задачу для системы уравнений Навье"=Стокса

\begin{equation}\label{eq:5.1}
    \begin{gathered}
        \frac{\partial v}{\partial t}+\sum_{i=1}^nv_i\frac{\partial v}{\partial x_i}-\nu\Delta v+\nabla p=f;
    \end{gathered}
\end{equation}

\begin{equation}\label{eq:5.2}
    \begin{gathered}
        \operatorname{div} v=0;
    \end{gathered}
\end{equation}

\begin{equation}\label{eq:5.3}
    \begin{gathered}
        v|_{t=0}=v_0;
    \end{gathered}
\end{equation}

\begin{equation}\label{eq:5.4}
    \begin{gathered}
        v|_{(0, T)\times\partial\Omega}=0.
    \end{gathered}
\end{equation}
            \end{minipage}
        \end{center}
    \end{frame}

    \begin{frame}
        \frametitle{Введение необходимых функциональных пространств}
        \begin{center}
            \begin{minipage}[h]{0.97\linewidth}
$L_p(\Omega)$ --- множество измеримых функций, суммируемых с $p$-ой степенью, где $1\le p< \infty$,
и нормой $\|v\|_{L_p(\Omega)}=(\int\limits_\Omega |v(x)|^pdx)^{1/p}$.

Пространство $L_{\infty}(\Omega)$ состоит из измеримых существенно ограниченных функций $v:\Omega\rightarrow R^n$. 
Функция $v:\Omega\rightarrow R^n$ называется существенной ограниченной, если существует число $C_1<\infty$, что $|v(x)|\le C_1$ при почти всех $x\in\Omega$.
Норма в $L_{\infty}(\Omega)$ задается
$\|v\|_{L_{\infty}(\Omega)}=ess \ \underset{x\in\Omega}{sup} |v(x)|$.

$W_p^m(\Omega)$ --- где $m\geqslant 1, \ p\geqslant 1$, пространство Соболева, состоящее из
функции, которые со своими обобщенными частными производными до порядка $m$ включительно принадлежат пространству $L_p(\Omega)$.\\
Норма в $W_p^m(\Omega)$ задается $\|v\|_{W_p^m(\Omega)}=\left(\sum\limits_{|\alpha|\leqslant m} \int\limits_\Omega |D^{\alpha}v(x)|^pdx\right)^{1/p}$.

$L_p(a,b;X)$ --- где $1\le p< \infty$ пространство суммируемых с $p$-ой степенью функций на $[a, b]$ со значениями в
банаховом пространстве $X$. Норма пространства $L_p(a, b;X)$ задается 
$\|v\|_{L_p(a, b;X)}=(\int\limits_0^T\|v(s)\|_X^p ds)^{1/p}.$

Через $L_{\infty}(a,b;X)$ будем обозначать множество всех измеримых существенно ограниченных функций $v:[a,b]\rightarrow X$.
Множество $L_{\infty}(a,b;X)$ является банаховым пространством относительно нормы $\|v\|_{L_{\infty}(a,b;X)}=ess \ \underset{x\in\Omega}{sup} \|v(s)\|_X$.
            \end{minipage}
        \end{center}
    \end{frame}

    \begin{frame}
        \frametitle{Введем определение сильного решения}
        \begin{center}
            \begin{minipage}[h]{0.97\linewidth}
\begin{definition}
    Сильным решением начально"=краевой задачи (\ref{eq:5.1})-(\ref{eq:5.4}) называется пара функций $v\in L_2(0, T; L_2(\Omega))$ и
    $p\in L_2(0, T; L_2(\Omega))$, удовлетворяющих следующим условиям:
\end{definition}

\begin{enumerate} 
    \item обобщенные частные производные функций, содержащихся в равенствах (\ref{eq:5.1})-(\ref{eq:5.2}), принадлежат пространству $L_2(0, T; L_2(\Omega))$;
    \item при подстановке функций уравнения (\ref{eq:5.1})-(\ref{eq:5.2}) обращаются в равенства в пространстве $L_2(0, T; L_2(\Omega))$;
    \item функция $v$ удовлетворяет начальному условию (\ref{eq:5.3}) и граничному условию (\ref{eq:5.4}).
\end{enumerate}
            \end{minipage}
        \end{center}
    \end{frame}

    \begin{frame}
        \frametitle{Актуальность проблемы}
        \begin{center}
            \begin{minipage}[h]{0.97\linewidth}

            \end{minipage}
        \end{center}
    \end{frame}

    \begin{frame}
        \frametitle{Актуальность проблемы}
        \begin{center}
            \begin{minipage}[h]{0.97\linewidth}

            \end{minipage}
        \end{center}
    \end{frame}

    % \begin{frame}
    %     \frametitle{Постановка задачи}
    %     \begin{center}
    %         \begin{minipage}[h]{0.97\linewidth}
    %             Для решения такой задачи необходимо на основе теории вибрационных машин и теоремы об оптимальности импульса Максвелла-Фейера
    %             разработать ПО для автоматизированного расчета характеристик импульсного погружателя
    %             с возможностью ввода начальных данных и наглядного вывода результатов\footnotemark[1]\footnotemark[2].
    %         \end{minipage}
    %     \end{center}
    %     \footnotetext[1]{\label{foot-1} Блехман И. И. Вибрационная механика. --- М. : Физико-математическая литература, 1994.}
    %     \footnotetext[2]{\label{foot-2} Костин Д. В. Бифуркация резонансных колебаний и оптимизация тригонометрического импульса по коэффициенту несимметрии // Математический сборник. --- М., 2016.}
    % \end{frame}


    % \begin{frame}
    %     \frametitle{Конструкция импульсного погружателя}
    %     \begin{center}
    %         \begin{minipage}[h]{0.97\linewidth}
    %             \begin{minipage}[h]{0.55\linewidth}
    %                 Работа погружателя основана на двух основных принципах:
    %                 \begin{enumerate} 
    %                     \item На эффекте резкого снижения сопротивлению погружения свайного элемента при сообщении последнему вибрации;
    %                     \item На действии полигармонического импульса, создаваемого центробежными силами системы дебалансов.
    %                 \end{enumerate}
    %                 При вращении валов (1) с дебалансами (2) на их ось крепления действует центробежная сила и погружатель получает вибрирующее движение,
    %                 которое через наголовник (3) сообщается свайному элементу (4).
    %             \end{minipage}
    %             % \hfill
    %             % \begin{minipage}[h]{0.36\linewidth}
    %             %     \begin{figure}[h]
    %             %         \centering
    %             %         \includegraphics[width=1\linewidth]{../img/scheme_porg_2.png}
    %             %         \caption{Схема импульсного погружателя.}
    %             %     \end{figure}
    %             % \end{minipage}
    %         \end{minipage}
    %     \end{center}
    % \end{frame}


    % \begin{frame}
    %     \frametitle{Конструкция дебаланса}
    %     \begin{center}
    %         \begin{minipage}[h]{0.97\linewidth}
    %             \begin{minipage}[h]{0.6\linewidth}
    %                 Пусть дан  дебаланс с радиусом $r$, радиус вала которого равен $R$,
    %                 $\omega$ --- угловая скорость и $l$ --- расстояние от центра масс до оси вращения дебаланса, а его масса будет равна $m$.
    %                 Центробежная сила:
    %                 \begin{equation}
    %                     \begin{gathered}
    %                         F_{\textrm{центр.}} = m \cdot \omega^2 \cdot l \\
    %                         \textrm{где } l = \frac{4 r}{3 \pi}
    %                     \end{gathered}
    %                 \end{equation}
    %                 \\
    %                 Гармонические колебания:
    %                 \begin{equation}
    %                     \begin{gathered}
    %                         x(t) = \lambda \cos (\omega t) \\
    %                         \textrm{где } \lambda = m \cdot \omega^2 \cdot l
    %                     \end{gathered}
    %                 \end{equation}
    %             \end{minipage}
    %             % \hfill
    %             % \begin{minipage}[h]{0.36\linewidth}
    %             %     \begin{figure}[h]
    %             %         \centering
    %             %         \includegraphics[width=1\linewidth]{../img/debalance.png}
    %             %         \caption{Схема дебаланса.}
    %             %     \end{figure}
    %             % \end{minipage}
    %         \end{minipage}
    %     \end{center}
    % \end{frame}


    % \begin{frame}
    %     \frametitle{Конструкция пары дебалансов}
    %     \begin{center}
    %         \begin{minipage}[h]{0.97\linewidth}
    %             Для компенсации горизонтальных сил в конструкции погружателя используются парные дебалансы.\\
    %             Гармонические колебания пары дебалансов:
    %             \begin{equation}
    %                 \begin{gathered}
    %                     x(t) = 2 \lambda \cos (\omega t), \textrm{где } \lambda = m \cdot \omega^2 \cdot l
    %                 \end{gathered}
    %             \end{equation}
    %             % \begin{figure}[h]
    %             %     \centering
    %             %     \includegraphics[width=0.52\linewidth]{../img/double_debalance.png}
    %             %     \caption{Схема пары дебалансов.}
    %             % \end{figure}
    %         \end{minipage}
    %     \end{center}
    % \end{frame}


    % \begin{frame}
    %     \frametitle{Гармонические колебания дебалансов}
    %     \begin{center}
    %         \begin{minipage}[h]{0.97\linewidth}
    %             При использовании нескольких пар дебалансов, вышестоящий уровень дебалансов должен иметь угловую скорость в два раза выше, чем прошлый.\\
    %             \newline
    %             \begin{minipage}[h]{0.44\linewidth}
    %                 % \begin{figure}[h]
    %                 %     \centering
    %                 %     $x(t) = \lambda_1 \cos (\omega t)$
    %                 %     \includegraphics[width=1\linewidth]{../grap/impulse_1.pdf}
    %                 %     \caption{Гармонические колебания для одной пары дебалансов.}
    %                 % \end{figure}
    %             \end{minipage}
    %             \hfill
    %             \begin{minipage}[h]{0.44\linewidth}
    %                 % \begin{figure}[h]
    %                 %     \centering
    %                 %     $x(t) = \lambda_1 \cos (\omega t) + \lambda_2 \cos (2\omega t)$
    %                 %     \includegraphics[width=1\linewidth]{../grap/impulse_2.pdf}
    %                 %     \caption{Гармонические колебания для двух пар дебалансов.}
    %                 % \end{figure}
    %             \end{minipage}
    %         \end{minipage}
    %     \end{center}
    % \end{frame}

    % \begin{frame}
    %     \frametitle{Гармонические нескольких пар дебалансов}
    %     \begin{center}
    %         \begin{minipage}[h]{0.97\linewidth}
    %             Гармоническое колебания для $n$ дебалансов, где $k$ --- порядковый номер пары дебалансов, будет иметь вид:
    %             \begin{equation}\label{eq:harmonic_sum}
    %                 \begin{gathered}
    %                     F = \sum\limits_{k = 1}^n 2 \lambda_k \cdot \cos (k \omega t), \lambda = m \cdot \omega^2 \cdot l
    %                 \end{gathered}
    %             \end{equation}
    %             Использование нескольких пар дебалансов разных характеристик позволяет увеличить импульс,
    %             направленный на погружение свайного элемента и уменьшить импульс, направленный в противоположную сторону.
    %         \end{minipage}
    %     \end{center}
    % \end{frame}


    % \begin{frame}
    %     \frametitle{Задача оптимизации}
    %     \begin{center}
    %         \begin{minipage}[h]{0.97\linewidth}
    %             Пусть $f_{\max}(t)$ --- максимальное значение импульса силы за время $t$, $f_{\min}(t)$ --- минимальное значение импульса за время $t$. Тогда:
    %             \begin{equation}
    %                 \begin{gathered}
    %                     K = \left| \frac{f_{\max}(t)}{f_{\min}(t)} \right| \rightarrow \max
    %                 \end{gathered}
    %             \end{equation}
    %             \begin{block}{Теорема\footnotemark[3]}\label{teorema}
    %                 Многочлен (\ref{eq:harmonic_sum}) является оптимальным т. и т. т. , когда он с точностью до постоянного множителя имеет вид суммы Фейера:
    %                 \begin{equation}\label{eq:feer}
    %                     \begin{gathered}
    %                         f_n(t) = \sum\limits_{k = 1}^n (n + 1 - k) \cos(kt)\\
    %                         \max \limits_{\lambda} K_n(\lambda) = n
    %                     \end{gathered}
    %                 \end{equation}
    %             \end{block}
    %         \end{minipage}
    %     \end{center}
    %     \footnotetext[3]{\label{foot-3} Костин Д. В. Бифуркация резонансных колебаний и оптимизация тригонометрического импульса по коэффициенту несимметрии
    %     // Математический сборник. --- М., 2016. — С. 90—109.}
    % \end{frame}

    % \begin{frame}
    %     \frametitle{Задача оптимизации}
    %     \begin{center}
    %         \begin{minipage}[h]{0.97\linewidth}
    %             Исходя из теоремы выше, следует, что:
    %             \begin{equation}
    %                 \begin{gathered}
    %                     \lambda_k = \frac{n - k + 1}{n} \cdot \lambda_1,\\
    %                     \textrm{где } \lambda_1 = m_1 \cdot \omega_{1}^{2} \cdot l_1
    %                 \end{gathered}
    %             \end{equation}
    %             Это позволяет найти коэффициент $\lambda_k$ для $k$-й пары дебалансов, когда общее количество дебалансов погружателя --- $n$.
    %         \end{minipage}
    %     \end{center}
    % \end{frame}

    % \begin{frame}
    %     \frametitle{Программная реализация}
    %     \begin{center}
    %         \begin{minipage}[h]{0.97\linewidth}
    %             При помощи применения теоремы об оптимальности модели полигармонического импульса и на основе теории вибрационных машин
    %             на языке Python была разработана программа для автоматического расчета характеристик дебалансов погружателя.
    %             % \begin{figure}[h]
    %             %     \centering
    %             %     \includegraphics[width=0.62\linewidth]{../img/xolm_2.png}
    %             %     \caption{Скриншот программы.}
    %             % \end{figure}
    %         \end{minipage}
    %     \end{center}
    % \end{frame}


    \begin{frame}
        \begin{alertblock}{}
            \centerline{\large Спасибо за внимание!}
        \end{alertblock}
    \end{frame}
\end{document}
